%
% حق نشر 1390-1402 دانش پژوهان ققنوس
% حقوق این اثر محفوظ است.
% 
% استفاده مجدد از متن و یا نتایج این اثر در هر شکل غیر قانونی است مگر اینکه متن حق
% نشر بالا در ابتدای تمامی مستندهای و یا برنامه‌های به دست آمده از این اثر
% بازنویسی شود. این کار باید برای تمامی مستندها، متنهای تبلیغاتی برنامه‌های
% کاربردی و سایر مواردی که از این اثر به دست می‌آید مندرج شده و در قسمت تقدیر از
% صاحب این اثر نام برده شود.
% 
% نام گروه دانش پژوهان ققنوس ممکن است در محصولات دست آمده شده از این اثر درج
% نشود که در این حالت با مطالبی که در بالا اورده شده در تضاد نیست. برای اطلاع
% بیشتر در مورد حق نشر آدرس زیر مراجعه کنید:
% 
% http://dpq.co.ir/licence
%
\chapter{ساختارهای مستند نویسی}

ایجاد مستند مناسب توسعه بر اساس سیستم‌های پیاده‌سازی شده یک چالش اساسی در توسعه
سیستم‌های نرم‌افزاری به حساب می‌آید. از این رو همواره گروه‌های زیادی در زمینه حل
مشکلات موجود در این زمینه تلاش کرده‌اند. نتیجه به دست آمده از تلاش‌های فراوان در
این زمینه انبوهی از روش‌ها و ابزارهای تولید مستند توسعه است که هر کدام ویژگی‌های
خاص خود را دارند.

   یکی از ابتدایی ترین نیازها، که در همان اوایل اختراع شبکه جهانی مطرح شد، ایجاد
یک استاندارد
   مناسب برای نمایش داده‌های موجود شبکه جهانی بود. از آن زمان تا کنون روش‌ها و
استاندارهای
   متفاوتی ایجاد شد. یکی از پر کاربرد ترین استانداردها استاندارد ابرمتن
\lr{HTML} است. استاندارد
   ابرمتن امروزه قالب‌ترین استاندارد مورد استفاده به روی شبکه جهانی است.

   با معرفی شدن زبان برنامه سازی جاوا، دسته‌ای از ابزارهای برنامه سازی نیز توسط
شرکت
   سان\LTRfootnote{Sun Micro System} معرفی شد. \lr{Java Doc} یکی از این ابزارها
   بود که
   در ایجاد مستند توسعه بر اساس کد منبع سیستم‌ها کاربرد داشت. با استفاده از این
ابزار فرایند
   ایجاد مستند بر اساس کد منبع سیستم، بسیار ساده شد تا جایی که امروزه این ابزار
به یکی از
   پرکاربرد ترین ابزار توسعه سیستم‌ها با استفاده از زبان برنامه سازی جاوا تبدیل
شده است.

%    این قسمت در مورد Qt Doc هست. زمانی که بخش مربوط به این بسته نوشته شد این
قسمت
%  	اضافه می‌شود.
% 	مصطفی برمشوری ۱۳۹۰
   گرچه بسته نرم‌افزاری \lr{Qt} تنهای یک بسته نرم‌افزاری برای ایجاد واسطه‌های
کاربری گرافیکی
   است اما به دلیل پرکاربرد بودن آن دسته‌ای از ابزارهای متفاوت برای حمایت از این
بسته نرم‌افزاری
   توسعه یافته اند. \lr{Qt Doc} یکی از این ابزارها است که در تولید مستند توسعه
به کار می‌رود.
   مستندهای توسعه این بسته نرم‌افزاری نیز به صورت کامل بر اساس استانداردهای
تعیین شده در
   ابزار \lr{Qt Doc} نوشته شده است.

    به مرور  و با هدف همگانی شدن ابزار \lr{Doxygen} استانداردهای یاد شده، به
استاندار مورد استفاده
    در \lr{Doxygen}  اضافه شده است. به بیان دیگر این ابزار قادر است که مستند
توسعه تمام کدهایی
    که بر اساس دیگر استانداردهای مستند نویسی ایجاد شده اند را، ایجاد کند. در این
فصل ابتدا نشان
    داده می‌شود که چگونه می‌توان با استفاده از استانداردهای ابر متن مستند توسعه
را نوشت. در ادامه
    چگونگی نوشتن مستند بر اساس استانداردهای تعیین شده در \lr{Java Doc} نیز بررسی
خواهد شد.

%  این مستند بر اساس دستورهای ابرمتن مورد استفاده در Doxygen ایجاد شده است. دسته بندی موجود برای این مستند
% بر اساس تجربه های خودم ایجاد شده است.
% مصطفی برمشوری ۱۳۹۰

\section{برچسب‌های ابرمتن}
با گسترش شبکه جهانی، ساختار متنی مورد استفاده در کاوشگرها جایگاه ویژه‌ای یافته است. از این رو امکان
استفاده از ساختار ابر متن در مستند نویسی یک امتیاز مهم به شمار می‌آید. در \lr{Doxygen} نیز امکان استفاده از برچسب‌های 
ابرمتن فراهم شده است. البته باید به این نکته توجه داشت که استفاده از برچسب‌های ابرمتن در نوشتن مستند به این معنی است که 
ابزارهایی مانند \lr{Doxygen} باید قادر به ترجمه برچسب‌ها به دیگر قالب‌های متنی مانند \lr{LaTex} باشند. گرچه توانایی ترجمه
برچسب‌های ابر متن به دیگر قالب‌های متنی در \lr{Doxygen} فراهم شده است اما  امکان ترجمه به تمام ساختارهای مورد حمایت محدودیت‌هایی
را در استفاده از این برچسب‌ها به دنبال داشته است.

در ساختار متنی ابر متن، امکان تعیین خصوصیت‌هایی برای هر برچسب وجود دارد. برای نمونه می‌توان  طول و عرض یک تصویر را در مستند 
تعیین کرد. در \lr{Doxygen} امکان ترجمه این خصوصیت‌ها به دیگر قالب‌های متنی مورد حمایت را فراهم نشده است. از این رو خصوصیت‌های 
تعیین شده در این برچسب‌ها تنها به خروجی ابر متن انتقال پیدا خواهد کرد و در دیگر ساختارهای مورد حمایت نادیده گرفته خواهد شد.

دسته‌ای از برچسب‌های مورد استفاده در ابرمتن، از نظر منطقی قابل استفاده در مستند نویسی نیست از این رو در \lr{Doxygen} انها را مورد 
حمایت قرار نمی‌گیرند. در این بخش تمام برچسب‌های مورد استفاده در ابر متن، بر اساس کاربرد‌ آنها مورد بررس قرار خواهد گرفت.

\subsection{برچسب‌هایی که حمایت نمی‌شود}
همان گونه که پیش از این بیان شده، دسته‌ای از برچسب‌های ابرمتن از نظر منطقی در نوشتن یک مستند قابل استفاده نمی‌باشد. برای نمونه
یکی از برچسب‌های مورد استفاده در ساختار ابرمتن \lr{BODY} است که بدنه اصلی متن هر صفحه را نشان می‌دهد. از انجا که بدنه متن
خروجی حتی در حالت ابرمتن تا زمان تولید کامل مستند مشخص نیست نمی‌توان این برچسب را به کار برد. برچسب‌های که در نوشتن مستند
مورد حمایت قرار نمی‌گیرد در فرایند ایجاد مستند نادیده گرفته خواهد شد. در زیر فهرستی از برچسب‌های ابرمتن که مورد حمایت قرار نمی‌گیرد 
آورده شده است:
\begin{itemize}
 \item \lr{BODY}
 \item \lr{FORM}
 \item \lr{META}
 \item \lr{MULTYICOL}
\end{itemize}

\begin{note}
 برای ایجاد یک مستند منظم بهتر است که از این برچسب‌ها به هیچ عنوان در نوشتن مستند استفاده نشود.
\end{note}

% این برچسب‌ها تنها در خروجی  ابرمتن ظاهر خواهد شد و در دیگر  خروجی ها ظاهر نمی ‌شود
% مصطفی برمشوری ۱۳۹۰
\subsection{برچسب‌های خاص}
دسته‌ای از برچسب‌های ابر متن تنها در خروجی ابرمتن از مستند‌ها ظاهر می‌شود  در حالی که در دیگر مستند‌ها نادیده گرفته می‌شود.
همان گونه که پیش از این نیز بیان شده دلیل عمده آن ابهام ایجاد شده توسط این برچسب‌ها در تبدیل مستند به دیگر قالب‌ها مستند است.
در زیر فهرست این برچسب‌ها آمده است.
\begin{itemize}
 \item \lr{IMG}
 \item \lr{DIV}
 \item \lr{SPAN}
\end{itemize}
ار برچسب \lr{img} برای ایجاد یک شکل استفاده می‌شود. ساختار کلی این برچسب به صورت زیر است.
\begin{latin}
\lstset{language=C++}  
\begin{lstlisting}[frame=single] 
    /**
     * <img src="path to pictuer" alt="Alternative text" />
     */
\end{lstlisting}
\end{latin}
که در آن \lr{src} مسیر شکل مورد نظر و \lr{alt} متن جایگزین را برای آن تعیین می‌کند. این برچسب تنها
منجر به ایجاد یک پیوند میان مستند و شکل مورد نظر خواهد شد، به بیان دیگر می‌توان این برچسب را به صورت
یک جاینگه‌دار برای یک شکل در نظر گرفت. اگر در مسیر تعیین شده برای شکل، پرونده شکل موجود نباشد و یا
به هر دلیلی کاوشگر مستند نتواند تصویر را نمایش دهد، متن جایگزین به نمایش در خواهد آمد.

برچسب \lr{div} یک بخش در مستند ایجاد می‌کند.  از این برچسب برای ایجاد یک دسته از موجودیت‌ها با یک
خصوصیت مشترک استفاده می‌شود، برای نمون کد زیر یک بخش را ایجاد می‌کند که تمام متن‌ها آن به رنگ سبز
است.
\begin{latin}
\lstset{language=C++}  
\begin{lstlisting}[frame=single] 
    /**
     * <div style="color:#00FF00">
     * 	<h3>Section</h3>
     * 		<p>Section text.</p>
     * 	</div>
     */
\end{lstlisting}
\end{latin}
معمولا کاوشگرها برای نمایش یک بخش ایجاد شده با این برچسب از یک خط تهی پیش و پس از بخش 
استفاده می‌کنند. در مستندها برای ایجاد لایه بندی در مستند از این برچسب به صورت گسترده استفاده
می‌شود.

از برچسب \lr{span} نیز برای ایجاد یک بخش در مستند استفاده می‌شود. گرچه با استفاده از این برچسب
همانند برچسب \lr{div} می‌توان یک بخش را در مستند ایجاد کرد اما برخلاف برچسب \lr{div} این برچسب
منجر به تغییر ساختار در مستند نمی‌شود. کد زیر یک واژه آبی رنگ را در یک متن ایجاد می‌کند.
\begin{latin}
\lstset{language=C++}  
\begin{lstlisting}[frame=single] 
    /**
     * <p> The car is 
     * 	<span style="color:#0000FF">blue</span>.
     * 	</p>
     */
\end{lstlisting}
\end{latin}
در مستندها از این برچسب برای ایجاد یک خصوصیت خاص در یک قسمت از متن استفاده می‌شود.


\subsection{دسته بندی}
در ساختار ابر متن از شش برچسب برای ایجاد بخش‌ها استفاده می‌شود. این برچسب‌ها که با استفاده از
یک شماره تعیین می‌شوند برای ایجاد ساختارهای سلسله مراتبی در متن مورد استفاده قرار می‌گیرند. در
\lr{Doxygen}
تنها از سه برچسب زیر در ایجاد مستند استفاده می‌شود.
\begin{itemize}
 \item \lr{H1}
 \item \lr{H2}
 \item \lr{H3}
\end{itemize}
در مستندها نیز از این برچسب‌ها برای ایجاد بخش و دسته استفاده می‌شود. به این نکته باید توجه داشت که
بخش‌های ایجاد شده با استفاده از این برچسب‌ها در شماره گذاری نادیده گرفته می‌شود. کد زیر چگونگی 
استفاده از این برچسب‌ها را نشان می‌دهد.
\begin{latin}
\lstset{language=C++}  
\begin{lstlisting}[frame=single] 
    /**
     * <h1> Section Title</h1>
     * <p> This is section text. </p>
     */
\end{lstlisting}
\end{latin}

\subsection{فهرست}
در ساختارهای ابر متن روش‌های متفاوتی برای ایجاد فهرست پیش بینی شده است. در حالت کلی سه
گونه فهرست وجود دارد که عبارت‌اند از:
\begin{itemize}
 \item ترتیب دار
 \item بدون ترتیب
 \item تعاریف
\end{itemize}
هر فهرست شامل چندین گزینه است که هر گزینه می‌تواند به نوبه خود یک فهرست دیگر باشد. توانایی
به کار بردن یک فهرست به عنوان یک گزینه در یک فهرست دیگر منجر به ایجاد یک ساختار سلسله مراتبی
در فهرست می‌شود. از آنجا که هیچ محدودیتی در ترکیب کردن فهرست‌ها وجود ندارد می‌توان گفت که 
چگونگی ترکیب آنها کاملا به کاربر بستگی دارد. در این بخش هر سه گونه فهرست را به صورت کوتاه بررسی
خواهیم کرد.
\subsubsection{ ترتیب دار و بدون ترتیب}
یک فهرست بدون ترتیب با برچسب \lr{UL} مشخص می‌شود. در این نوع فهرست گزینه‌ها بدون شماره ظاهر
می‌شوند  از همین رو این نوع فهرست را بدون ترتیب می‌نامند. گزینه‌های هر فهرست نیز با برچسب \lr{LI} مشخص
می‌شود. در کد زیر یک فهرست بدون ترتیب ایجاد شده است.
\begin{latin}
\lstset{language=C++}  
\begin{lstlisting}[frame=single] 
    /**
     *	<UL>
     * 		<LI>Item</LI>
     * 		<LI>Item</LI>
     * 	</UL>
     */
\end{lstlisting}
\end{latin}
برخلاف فهرست بدون ترتیب، که در آن گزینه‌ها بدون شماره ظاهر می‌شوند، در فهرست ترتیب دار تمام گزینه‌ها با 
یک شماره شروع می‌شوند. شماره گذاری گزینه‌ها از عدد یک شروع می‌شود. گزینه‌ها در فهرست ترتیب‌دار همانند
فهرست‌های بدون ترتیب مشخص می‌شوند. در نمونه کد زیر یک فهرست ترتیب دار با استفاده از دو گزینه ایجاد
شده است.
\begin{latin}
\lstset{language=C++}  
\begin{lstlisting}[frame=single] 
    /**
     *	<OL>
     * 		<LI>Item</LI>
     * 		<LI>Item</LI>
     * 	</OL>
     */
\end{lstlisting}
\end{latin}

\subsubsection{تعاریف}
فهرست تعاریف گردایه‌ای از گزینه‌ها است که به دنبال هر کدام از گزینه‌ها یک تعریف قرار می‌گیرد.
برای ایجاد یک فهرست تعاریف از برچسب \lr{DL} استفاده می‌شود. در هر فهرست تعاریف دو
موجودیت وجود دارد که عبارت‌اند از عبارت و تعریف آن. در فهرست تعاریف عبارت‌ها با استفاده
از برچسب \lr{DT} و هر تعریف با استفاده از برچسب \lr{DD} مشخص می‌شود. در کد زیر 
یک نمونه از فهرست تعاریف ایجاد شده است.
\begin{latin}
\lstset{language=C++}  
\begin{lstlisting}[frame=single] 
    /**
     *	<DL>
     * 		<DT>Item</DT>
     * 		<DD>This is first item.</DD>
     * 		<DT>Item</DT>
     * 		<DD>This is second item.</DD>
     * 	</UL>
     */
\end{lstlisting}
\end{latin}

\subsection{جدول}
جدول ساختاری است که امکان ساماندهی کردن داده‌ها را (مانند تصویر، متن، پیوند، فهرست و غیره)
به صورت سطری و ستونی، فراهم می‌کند. برای هر جدول می‌توان یک توصیف کوتاه در نظر گرفت، که 
در اطراف جدول ظاهر می‌شود. جدول با استفاده از برچسب \lr{TABLE}، در مستندهای ابر متن ایجاد
می‌شود. داده‌ها با استفاده از سطرهایی ساماندهی می‌شوند که هر کدام شامل چندین خانه است. در جدول هر سطر 
با استفاده از \lr{TR} و هر خانه موجود در سطر با استفاده از \lr{TD} مشخص می‌شود. کد زیر یک نمونه
جدول را که دارای دو سطر و دو ستون است را ایجاد می‌کند.
\begin{latin}
\lstset{language=C++}  
\begin{lstlisting}[frame=single] 
    /**
     *	<TABLE>
     * 		<TR><TD>ITEM</TD>
     * 			<TD>ITEM</TD></TR>
     * 		<TR><TD>ITEM</TD>
     * 			<TD>ITEM</TD></TR>
     * 	</TABLE>
     */
\end{lstlisting}
\end{latin}
همان گونه که اشاره شد، می‌توان برای هر جدول یک توصیف در نظر گرفت. توصیف یک جدول با استفاده از
 برچسب \lr{CAPTION} مشخص می‌شود که داده‌های مورد نیاز در توصیف جدول را در بر می‌گیرد.
 توصیف جدول می‌تواند شامل تمام برچسب‌ها و دادهای ممکن نیز باشد. این توصیف به صورت پیش‌فرض
 در بالای جدول قرار می‌گیرد. در کد زیر یک توصیف کوتاه به جدول اضافه شده است.
\begin{latin}
\lstset{language=C++}  
\begin{lstlisting}[frame=single] 
    /**
     *	<TABLE>
     * 		<CAPTION>This is a simple table with a row</CAPTION>
     * 		<TR><TD>ITEM</TD>
     * 			<TD>ITEM</TD></TR>
     * 	</TABLE>
     */
\end{lstlisting}
\end{latin}

\subsection{قالب بندی متن}
دسته‌ای از برچسب‌های ابر متن وجود دارد که تنها برای ساختاردهی کردن به متن نوشته شده به کار 
می‌روند. این دسته از برچسب‌ها را برچسب‌های قالب بندی می‌نامیم. در جدول 
\ref{جدول_برچسبهای_قالببندی}
فهرست کاملی از برچسب‌های قالب بندی  آروده شده است.
\begin{table}[ht]
 \centering
  {%
    \newcommand{\mc}[3]{\multicolumn{#1}{#2}{#3}}
    \begin{center}
    \begin{tabular}{|l|l|}\hline
      \mc{1}{r}{\lr{CENTER}} & \mc{1}{r}{متن را در مرکز صفحه قرار می‌دهد}\\\hline
      \mc{1}{r}{\lr{BR}} & \mc{1}{r}{یک سطر جدید ایجاد می‌کند}\\\hline
      \mc{1}{r}{\lr{CODE}} & \mc{1}{r}{کد برنامه نویسی را مشخص می‌کند}\\\hline
      \mc{1}{r}{\lr{DFN}} & \mc{1}{r}{یک عبارت را به وجود می‌آورد}\\\hline
      \mc{1}{r}{\lr{EM}} & \mc{1}{r}{یک متن تاکید شده}\\\hline
      \mc{1}{r}{\lr{HR}} & \mc{1}{r}{ترسیم یک خط افقی}\\\hline
      \mc{1}{r}{\lr{P}} & \mc{1}{r}{یک پاراگراف را تعیین می‌کند}\\\hline
      \mc{1}{r}{\lr{I}} & \mc{1}{r}{متن خمیده ایجاد می‌کند}\\\hline
      \mc{1}{r}{\lr{PRE}} & \mc{1}{r}{متن به همان گونه که نشته شده ظاهر می‌شود}\\\hline
      \mc{1}{r}{\lr{SMAL}} & \mc{1}{r}{متن با اندازه کوچک ایجاد می‌شود}\\\hline
      \mc{1}{r}{\lr{STRONG}} & \mc{1}{r}{متن با اندازه بزرگ ایجاد می‌شود}\\\hline
      \mc{1}{r}{\lr{SUB}} & \mc{1}{r}{متن پایین نویس را تعیین می‌کند}\\\hline
      \mc{1}{r}{\lr{SUP}} & \mc{1}{r}{متن بالا نویس را تعیین می کند}\\\hline
      \mc{1}{r}{\lr{TT}} & \mc{1}{r}{متن را با استفاده از قلم تلتایپ ترسیم می‌کند}\\\hline
      \mc{1}{r}{\lr{VAR}} & \mc{1}{r}{متغیرها را به صورت خاص نشان می‌دهد}\\\hline
    \end{tabular}
    \end{center}
  }%
 \caption{فهرست برچسب‌های مورد استفاده در ساختار دهی کردن مستندها. }
 \label{جدول_برچسبهای_قالببندی}
\end{table}

\subsection{پیوند}
همان‌گونه که پیش از این گفته شد، تمام خصوصیت‌های تعیین شد در مستند برای برچسب‌ها
توسط \lr{Doxygen} نادیده گرفته می‌شود اما در این میان برچسب \lr{a} یک استثنا است. می‌توان از دو
خصوصیت \lr{NAME} و \lr{HREF}  همراه با این برچسب استفاده کرد. مهمترین خصوصت این 
برچسب \lr{HREF} است که با استفاده از آن آدرس مستندهای دیگر برای ایجاد پیوند تعیین می‌شود.
کارکرد خصوصیت \lr{NAME} نیز مانند برچسب \lr{HREF} است با این تفاوت که در این 
خصوصیت از نام بخش به عنوان آدرس در  پیوند استفاده می‌شود. کد زیر یک نمونه از ایجاد پیوند را در
مستند نشان می‌دهد.
\begin{latin}
\lstset{language=C++}  
\begin{lstlisting}[frame=single] 
    /**
     * <a href="http://www.p-simorgh.com"> click for more information</a>
     */
\end{lstlisting}
\end{latin}

% \subsection{واژک‌های ویژه}
% در این بخش واژکهای ویژه مورد استفاده در ابرمتن و doxygen بررسی می شود
%     &copy; the copyright symbol
%     &tm; the trade mark symbol
%     &reg; the registered trade mark symbol
%     &lt; less-than symbol
%     &gt; greater-than symbol
%     &amp; ampersand
%     &apos; single quotation mark (straight)
%     &quot; double quotation mark (straight)
%     &lsquo; left single quotation mark
%     &rsquo; right single quotation mark
%     &ldquo; left double quotation mark
%     &rdquo; right double quotation mark
%     &ndash; n-dash (for numeric ranges, eg. 2–8)
%     &mdash; m-dash (for parenthetical punctuation — like this)
%     &?uml; where ? is one of {A,E,I,O,U,Y,a,e,i,o,u,y}, writes a character with a diaeresis accent (like ä).
%     &?acute; where ? is one of {A,E,I,O,U,Y,a,e,i,o,u,y}, writes a character with a acute accent (like á).
%     &?grave; where ? is one of {A,E,I,O,U,a,e,i,o,u,y}, writes a character with a grave accent (like à).
%     &?circ; where ? is one of {A,E,I,O,U,a,e,i,o,u,y}, writes a character with a circumflex accent (like â).
%     &?tilde; where ? is one of {A,N,O,a,n,o}, writes a character with a tilde accent (like ã).
%     &szlig; write a sharp s (i.e. ß) to the output.
%     &?cedil; where ? is one of {c,C}, writes a c-cedille (like ç).
%     &?ring; where ? is one of {a,A}, writes an a with a ring (like å).
%     &nbsp; a non breakable space.

\subsection{نوشتن یک مستند} 
شاید این خنده دار باشد که فردی برای یک مستند، مستند دیگری بنویسد، با این حال در عمل نیاز به نوشتن
مستندهای کوتاه برای مستندهای ایجاد شده است. مستندهای نوشته شده با استفاده از برچسب مستند 
توسط \lr{Doxygen} نادیده گرفته می‌شود. ساختار کلی این برچسب به صورت زیر است.
\begin{latin}
\lstset{language=C++}  
\begin{lstlisting}[frame=single] 
    /**
     * <!--
     * 		this is simple comment
     *  -->
     */
\end{lstlisting}
\end{latin}
%
% حق نشر 1390-1402 دانش پژوهان ققنوس
% حقوق این اثر محفوظ است.
% 
% استفاده مجدد از متن و یا نتایج این اثر در هر شکل غیر قانونی است مگر اینکه متن حق
% نشر بالا در ابتدای تمامی مستندهای و یا برنامه‌های به دست آمده از این اثر
% بازنویسی شود. این کار باید برای تمامی مستندها، متنهای تبلیغاتی برنامه‌های
% کاربردی و سایر مواردی که از این اثر به دست می‌آید مندرج شده و در قسمت تقدیر از
% صاحب این اثر نام برده شود.
% 
% نام گروه دانش پژوهان ققنوس ممکن است در محصولات دست آمده شده از این اثر درج
% نشود که در این حالت با مطالبی که در بالا اورده شده در تضاد نیست. برای اطلاع
% بیشتر در مورد حق نشر آدرس زیر مراجعه کنید:
% 
% http://dpq.co.ir/licence
%
% در این مستند به ابزارها و روشهای نوشتن مستند پرداخته می‌شود که از مستند سازی java
% وارد Doxygen شده است.
 
\section{تولیدگر مستند جاوا}
امروزه ابزارها و روش‌های متفاوتی برای تولید خودکار مستندهای توسعه به وجود آمده‌اند. یکی از پرکاربرد ترین ابزار تولید مستند توسعه، \lr{Java Doc} است\cite{javadocsun}. مهمترین ابزار مورد استفاده در ایجاد مستند هسته جاوا و بسیاری از بسته‌های 
نرم‌افزاری همین ابزار است. کاربرد گسترده این ابزار تا جایی است که در بسیاری از موارد تنها با استفاده از استاندارهای تعریف شده
در آن مستند بسته‌های نرم‌افزاری توسعه یافته است.

کاربرد این ابزار تا جایی پیش رفته است که نه تنها در تولید مستند از کد منبع، بلکه در تولید مستند بر اساس مدل‌های ایجاد شده
نیز به کار برده می‌شود. در بسیاری از موارد توسعه مدل هم گام با توسعه نرم‌افزار پیش می‌رود، از این رو می‌توان مدل را نیز در 
قالب کد منبع توسعه داد. امروزه روش‌ها و قراردادهای متفاوتی در زمینه ایجاد و تعریف مدل در متن برنامه و ایجاد مستند بر
اساس ابزار تولید مستند \lr{JavaDoc} توسعه یافته است\cite{Kramer:1999:ADS:318372.318577}.

نه تنها سادگی در به کارگیری این ابزار، بلکه توانایی‌های این نرم افزار در برخورد با چالش‌های موجود در تولید مستند توسعه، منجر به استفاده گسترده از این ابزار در توسعه مستند‌ها شده است\cite{Leslie:2002:UJX:584955.584971}. کاربردهای متفاوت و
سادگی به کارگیری این ابزار، آن را به عنوان یک ابزار پایه در برنامه سازی با جاوا تبدیل کرده به گونه‌ای که در کتاب‌های برنامه سازی
و درسی جاوا همواره به آن پرداخته می‌شود\cite{Schildt:2000:JCR:557816}.

برچسب‌ها در \lr{Java Doc} با استفاده از نشانه \lr{@} مشخص می‌شود در حالی که قرارداد \lr{Doxygen} 
استفاده از علامت اسلش\footnote{Slash} است. برای نمونه، در \lr{Java Doc} خروجی یک متد به صورت زیر تعیین
می‌شود.
\begin{latin}
\lstset{language=C++}  
\begin{lstlisting}[frame=single]
  /**
   * @return <document>
   */
\end{lstlisting}
\end{latin}
گرچه قرارداد پیش‌فرض در \lr{Doxygen} استفاده از نشانه اسلش است اما،  امکان تعیین برچسب‌ها به روش
مورد استفاده در \lr{Java Doc} نیز فراهم شده است. در \lr{Doxygen} تمام برچسب‌های استاندارد
موجود در \lr{Java Doc} قابل استفاده است، از این رو می‌توان با استفاده از این ابزار مستند توسعه را برای کدهایی ایجاد
کرد که بر اساس قرارداد موجود در \lr{Java Doc} مستند شده‌اند.

\lr{Java Doc}، در فرایند تولید مستند توسعه، برچسب‌هایی تعریف نشده را ندیده می‌گیرد و آنها را در خروجی وارد نمی‌کند از این
رو  نوشتن برچسب‌ها بر اساس قراردادهای \lr{Java Doc} یک مزیت به شمار می‌آید. مستند‌های که به این روش نوشته می‌شوند نه
تنها می‌توانند در \lr{Java Doc} به کار گرفته شوند بلکه در \lr{Doxygen} نیز قابل استفاده هستند.
%
% حق نشر 1390-1402 دانش پژوهان ققنوس
% حقوق این اثر محفوظ است.
% 
% استفاده مجدد از متن و یا نتایج این اثر در هر شکل غیر قانونی است مگر اینکه متن حق
% نشر بالا در ابتدای تمامی مستندهای و یا برنامه‌های به دست آمده از این اثر
% بازنویسی شود. این کار باید برای تمامی مستندها، متنهای تبلیغاتی برنامه‌های
% کاربردی و سایر مواردی که از این اثر به دست می‌آید مندرج شده و در قسمت تقدیر از
% صاحب این اثر نام برده شود.
% 
% نام گروه دانش پژوهان ققنوس ممکن است در محصولات دست آمده شده از این اثر درج
% نشود که در این حالت با مطالبی که در بالا اورده شده در تضاد نیست. برای اطلاع
% بیشتر در مورد حق نشر آدرس زیر مراجعه کنید:
% 
% http://dpq.co.ir/licence
%
% در این مستند به ابزارها و روشهای نوشتن مستند پرداخته می‌شود که از مستند سازی Qt
% وارد Doxygen شده است.
\section{مستندگر کیوتی}

\lr{QtDoc} یا \lr{QDoc} ابزاری جهت ایجاد مستند توسعه بر اساس متن برنامه است که توسط توسعه دهندگان
بسته نرم‌افزاری \lr{Qt} مورد استفاده قرار می‌گیرد. این ابزار بر اساس مستند‌های نوشته شده در 
پرونده‌های \lr{.cpp} و \lr{.qdoc}،  مستند توسعه را در قالب‌هایی مانند \lr{XML} و یا دیگر قالب‌های
متنی ایجاد می‌کند. توجه به این نکته مهم است که، در این نرم‌افزار پرونده‌های سرایند\footnote{\lr{Header Files}}
در فرآیند استخراج متن نادیده گرفته می‌شوند. متن‌های مستند در \lr{QDoc} با استفاده از نشانه تعجب 
در متن برنامه مشخص می‌شود. در نمونه زیر ساختار نوشتن یک مستند بر اساس استاندارد \lr{QDoc}
نشان داده شده است.
\begin{latin}
 \lstset{language=C++}  
\begin{lstlisting}[frame=single] 
  /*! 
      \class QObject
      \brife QObject is base class in Qt.
      
      QObject is heart of the Qt lib.
   */
\end{lstlisting}
\end{latin}

همان گونه که در نمونه بالا قابل مشاهده است، برچسب‌ها در استاندارد \lr{QDoc} همانند 
برچسب‌ها در \lr{Doxygen}  با استفاده از \textbackslash مشخص می‌شوند. تمام برچسب‌های
موجود در استاندارد مستند نویسی \lr{QDoc} را می‌توان به سه دسته تقسیم کرد، که عبارت‌اند از:
\begin{itemize}
 \item موضوع
 \item متن
 \item نشانک‌ها
\end{itemize}
تمام برچسب‌های تعریف شده در استاندارد \lr{QDoc} ، در \lr{Doxygen} نیز تعریف شده است، از این رو 
با استفاده از \lr{Doxygen} می‌توان مستند توسعه را بر اساس متن برنامه‌های ایجاد کرد که بر اساس
استانداردهای \lr{QDoc}‌ایجاد شده اند.در این بخش به صورت گذرا، هرکدام از این دسته‌ها را بررسی 
خوا‌هیم کرد.
\subsection{موضوع}
برچسب‌های موضوع تعیین می‌کنند که یک مستند نوشته شده به کدام قسمت از متن برنامه 
مربوط می‌شود. دسته‌ای از این برچسب‌ها نیز برای ایجاد صفحه‌هایی از مستند‌ها و دسته بندی
کردن آنها به کار می‌روند. زمانی که با استفاده از برچسب‌های موضوع تعیین نمی‌شود که مستند
به کدام قسمت از متن مربوط می‌شود، مستند به کدی نسبت داده می‌شود که بلافاصله بعد از 
مستند قرار گرفته است. اگر برچسب موضوع نتواند به درستی قسمتی از متن برنامه را آدرس دهی
کند، برچسب نادیده گرفته می‌شود و مستند به متن برنامه‌ای که بلافاصله بعد از آن آمده است 
نسبت داده می‌شود. در جدول \ref{جدول_برچسب_عنوان_QDoc} فهرستی از پرکاربرد ترین
برچسب‌های موضوع آورده شده است.
\begin{table}[ht]
 \centering
  {%
    \newcommand{\mc}[3]{\multicolumn{#1}{#2}{#3}}
    \begin{center}
    \begin{tabular}{|l|l|}\hline
      \mc{1}{r}{\lr{class}} & \mc{1}{r}{x}\\\hline
      \mc{1}{r}{\lr{enum}} & \mc{1}{r}{x}\\\hline
      \mc{1}{r}{\lr{example}} & \mc{1}{r}{x}\\\hline
      \mc{1}{r}{\lr{externalpage}} & \mc{1}{r}{x}\\\hline
      \mc{1}{r}{\lr{fn (function)}} & \mc{1}{r}{x}\\\hline
      \mc{1}{r}{\lr{group}} & \mc{1}{r}{x}\\\hline
      \mc{1}{r}{\lr{headerfile}} & \mc{1}{r}{x}\\\hline
      \mc{1}{r}{\lr{macro}} & \mc{1}{r}{x}\\\hline
      \mc{1}{r}{\lr{module}} & \mc{1}{r}{x}\\\hline
      \mc{1}{r}{\lr{namespace}} & \mc{1}{r}{x}\\\hline
      \mc{1}{r}{\lr{page}} & \mc{1}{r}{x}\\\hline
      \mc{1}{r}{\lr{property}} & \mc{1}{r}{x}\\\hline
      \mc{1}{r}{\lr{service}} & \mc{1}{r}{x}\\\hline
      \mc{1}{r}{\lr{typedef}} & \mc{1}{r}{x}\\\hline
      \mc{1}{r}{\lr{variable}} & \mc{1}{r}{x}\\\hline
    \end{tabular}
    \end{center}
  }%
 \caption{فهرست پر کاربرد ترین برچسب‌های عنوان در استاندارد \lr{QDoc}. }
 \label{جدول_برچسب_عنوان_QDoc}
\end{table}

\subsection{متن}
مستندهای متن، دسته‌ای از مستندها هستند که در ابزارهای خودکار مستند سازی قادر به تشخیص و ایجاد آنها نیستند.
برای نمون ابزارهای مستند ساز قادر به تشخیص مطلب‌های مرتبط با هم نیست. از این رو دسته‌ای از برچسب‌ها 
پیش بینی شده‌اند که در ایجاد و مدیریت اینگونه مستندها به کار می‌روند. به این دسته از برچسب‌ها برچسب‌های متن
گفته می‌شود. در جدول \ref{جدول_برچسب_متن_QDoc} فهرستی از پرکابرد ترین برچسب‌های متن آمده است.
\begin{table}[ht]
 \centering
  {%
    \newcommand{\mc}[3]{\multicolumn{#1}{#2}{#3}}
    \begin{center}
    \begin{tabular}{|l|l|}\hline
      \mc{1}{r}{\lr{compat}} & \mc{1}{r}{x}\\\hline
      \mc{1}{r}{\lr{contentspage}} & \mc{1}{r}{x}\\\hline
      \mc{1}{r}{\lr{indexpage}} & \mc{1}{r}{x}\\\hline
      \mc{1}{r}{\lr{ingroup}} & \mc{1}{r}{x}\\\hline
      \mc{1}{r}{\lr{inherits}} & \mc{1}{r}{x}\\\hline
      \mc{1}{r}{\lr{inmodule}} & \mc{1}{r}{x}\\\hline
      \mc{1}{r}{\lr{internal}} & \mc{1}{r}{x}\\\hline
      \mc{1}{r}{\lr{mainclass}} & \mc{1}{r}{x}\\\hline
      \mc{1}{r}{\lr{nextpage}} & \mc{1}{r}{x}\\\hline
      \mc{1}{r}{\lr{nonreentrant}} & \mc{1}{r}{x}\\\hline
      \mc{1}{r}{\lr{obsolete}} & \mc{1}{r}{x}\\\hline
      \mc{1}{r}{\lr{overload}} & \mc{1}{r}{x}\\\hline
      \mc{1}{r}{\lr{previouspage}} & \mc{1}{r}{x}\\\hline
      \mc{1}{r}{\lr{relates}} & \mc{1}{r}{x}\\\hline
      \mc{1}{r}{\lr{since}} & \mc{1}{r}{x}\\\hline
      \mc{1}{r}{\lr{startpage}} & \mc{1}{r}{x}\\\hline
      \mc{1}{r}{\lr{subtitle}} & \mc{1}{r}{x}\\\hline
      \mc{1}{r}{\lr{title}} & \mc{1}{r}{x}\\\hline
    \end{tabular}
    \end{center}
  }%
 \caption{فهرست پر کاربرد ترین برچسب‌های متن در استاندارد \lr{QDoc}. }
 \label{جدول_برچسب_متن_QDoc}
\end{table}


\subsection{نشانک‌ها}
آخرین دسته از برچسب‌ها، نشانک‌ها هستند. نشانک‌ها برای نوشتن و ساختاردهی کردن متن ایجاد شده
مورد استفاده قرار می‌گیرند. برای نمونه فهرست، جدول، پیونده به دیگر مستندها، ویا استفاده از 
یک تصویر خاص با استفاده از برچسب‌های نشانک ایجاد می‌شوند. در جدول \ref{جدول_برچسب_نشانک_QDoc}
فهرستی از پرکاربرد ترین نشانک‌ها در استاندارد \lr{QDoc} آمده است.
\begin{table}[ht]
 \centering
  {%
    \newcommand{\mc}[3]{\multicolumn{#1}{#2}{#3}}
    \begin{center}
    \begin{tabular}{|l|l|}\hline
      \mc{1}{r}{\lr{a}} & \mc{1}{r}{x}\\\hline
      \mc{1}{r}{\lr{bold}} & \mc{1}{r}{x}\\\hline
      \mc{1}{r}{\lr{brief}} & \mc{1}{r}{x}\\\hline
      \mc{1}{r}{\lr{c}} & \mc{1}{r}{x}\\\hline
      \mc{1}{r}{\lr{caption}} & \mc{1}{r}{x}\\\hline
      \mc{1}{r}{\lr{chapter}} & \mc{1}{r}{x}\\\hline
      \mc{1}{r}{\lr{code}} & \mc{1}{r}{x}\\\hline
      \mc{1}{r}{\lr{footnote}} & \mc{1}{r}{x}\\\hline
      \mc{1}{r}{\lr{image}} & \mc{1}{r}{x}\\\hline
      \mc{1}{r}{\lr{include}} & \mc{1}{r}{x}\\\hline
      \mc{1}{r}{\lr{input}} & \mc{1}{r}{x}\\\hline
      \mc{1}{r}{\lr{part}} & \mc{1}{r}{x}\\\hline
      \mc{1}{r}{\lr{quotation}} & \mc{1}{r}{x}\\\hline
      \mc{1}{r}{\lr{section1}} & \mc{1}{r}{x}\\\hline
      \mc{1}{r}{\lr{section2}} & \mc{1}{r}{x}\\\hline
      \mc{1}{r}{\lr{section3}} & \mc{1}{r}{x}\\\hline
      \mc{1}{r}{\lr{section4}} & \mc{1}{r}{x}\\\hline
      \mc{1}{r}{\lr{underline}} & \mc{1}{r}{x}\\\hline
    \end{tabular}
    \end{center}
  }%
 \caption{فهرست پر کاربرد ترین برچسب‌های نشانک در استاندارد \lr{QDoc}. }
 \label{جدول_برچسب_نشانک_QDoc}
\end{table} 
