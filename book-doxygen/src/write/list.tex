% این مستند در مورد ایجاد فهرست در Doxygen است. 
% مصطفی برمشوری ۱۳۹۰
\chapter{ایجاد فهرست}

\lr{Doxygen} روشهای متفاوتی را برای ایجاد یک فهرست در مستندها فراهم کرده است.
روشهای تعیین شده در \lr{Doxygen} را می توان به سه دسته تقسیم کرد: روشهای داخلی،
استفاده از برچسبهای \lr{HTM} و استفاده از برچسبهای تعریف شده در دیگر ابزارهای
مستند سازی . گرچه روشهای داخلی تعریف شده در \lr{Doxygen} روش مناسبی را برای
ایجاد فهرست فراهم کرده‌اند اما استفاده از برچسبهای \lr{HTML} برتریهایی نسبت به
این روشها دارد که در ادامه به آنها خواهیم پرداخت.

از آنجا که یکی از هدف‌های \lr{Doxygen} توانایی تولید مستند بر اساس دیگر
استاندارهای مستند سازی است، برچسبهای مورد استفاده در دیگر ابزارهای مستند سازی
نیز به صورت داخلی تعریف شده و قابل استفاده می باشد. از این رو نه تنها می‌توان از
روش‌های ایجاد شده در دیگر ابزارها استفاد کرد بلکه مستند‌های ایجاد شده مبتنی بر
استانداردهای دیگر را نیز مورد استفاده قرار داد.

\section{روش داخلی}

در این روش هر گزینه از فهرست با استفاده از یک خط تیره (علامت منفی) در ابتدای سطر
تعیین می‌شود. زمانی که نیاز به ایجاد یک گزینه شماره دار باشد بعد از خط تیره
علامت عدد (?) نیز قرار می‌گیرد. برای ایجاد زیر فهرست برای یک گزینه باید گزینه
های جدید را با استفاده از فضای خالی جلوتر از گزینه مورد نظر قرار داد.
در زیر یک نمونه فهرست ایجاد شده به این روش آورده شده است.

\begin{latin}
\lstset{language=C++}  
\begin{lstlisting}[frame=single] 
/**
 *  A list of events:
 *    - mouse events
 *         -# mouse move event
 *         -# mouse click event\n
 *            More info about the click event.
 *         -# mouse double click event
 *    - keyboard events
 *         -# key down event
 *         -# key up event
 *
 *  More text here.
 */
\end{lstlisting}
\end{latin}

در نهایت زمانی که با استفاده از \lr{Doxygen} این مستند به قالبهای دیگر تبدیل شود
نتیجه‌ای مشابه به نوشته زیر به دست خواهد آمد.

 \begin{latin}
    \begin{itemize}
	\item mouse events
		\begin{itemize}
			\item  mouse move event
			\item  mouse click event
			\item   More info about the click event.
			\item   mouse double click event
		\end{itemize}
	\item keyboard events
		\begin{itemize}
			\item  key down event
			\item  key up event
		\end{itemize}
  \end{itemize}
\end{latin}

\begin{note}
 تعیین  سطح جدید از فهرست نیاز به ایجاد فضای خالی به اندازه یک \lr{TAB} در
 ابتدای هر گزینه دارد اما ممکن است که اندازه \lr{TAB} در هر مستند متفاوت باشد.
 از این رو پیش از هر چیز از این نکته اطمینان حاصل کنید که به چه اندازه باید فضای
 خالی استفاده شود. میزان فضای خالی مورد استفاده برای هر \lr{TAB} در پرونده
 تنظیمات موجود است. این اندازه به صورت پیشفرض برابر با هشت فضای خالی در نظر
 گرفته می شود.
\end{note}

برای تعیین انتهای یک فهرست (در هر سطح) به دو روش می توان عمل کرد. در روش نخست یک
سطر خالی بعد از گزینه نوشته می‌شود. این روش زمانی مناسب است که یک فهرست به صورت
کامل تمام شده است و بعد از آن متنهای دیگری از مستند قرار خواهد گرفت. در روش دوم
در یک سطر جدید یک نقطه گذاشته می شود.
این روش برای نشان دادن انتهای فهرستی که دارای چندین سطح می‌باشند مناسب است. در
نمونه زیر  چگونگی استفاده از نقطه برای تعیین انتهای یک فهرست نشان داده شده است.

\begin{latin}
\lstset{language=C++}  
\begin{lstlisting}[frame=single] 
/**
 * Text before the list
 * - list item 1
 *   - sub item 1
 *     - sub sub item 1
 *     - sub sub item 2
 *     . 
 *     The dot above ends the sub sub item list.
 *     More text for the first sub item
 *   .
 *   The dot above ends the first sub item.
 *   More text for the first list item
 *   - sub item 2
 *   - sub item 3
 * - list item 2
 * .
 * More text in the same paragraph.
 *
 * More text in a new paragraph.
 */
\end{lstlisting}
\end{latin}

\section{برچسب‌های ابر متن}

یک روش ساده و بسیار مناسب برای ایجاد فهرست استفاده از برچسب‌های \lr{HTML} است. 
برتری مهم این روش نسبت به روش‌های دیگر، امکان استفاده از چندین پاراگراف در نوشتن
یک گزینه است.  در زیر نمونه مطرح شده در قسمت پیشین با استفاده از برچسب‌های
\lr{HTML} باز نویسی شده است.

\begin{latin}
\lstset{language=C++}  
\begin{lstlisting}[frame=single] 
  /*! 
   *  A list of events:
   *  <ul>
   *  <li> mouse events
   *     <ol>
   *     <li>mouse move event
   *     <li>mouse click event\n
   *         More info about the click event.
   *     <li>mouse double click event
   *     </ol>
   *  <li> keyboard events
   *     <ol>     
   *     <li>key down event
   *     <li>key up event
   *     </ol>
   *  </ul>
   *  More text here.
   */
\end{lstlisting}
\end{latin}

از انجا که فهرست‌هایی که در سطح‌های پایین‌تر قرار می‌گیرند با استفاده از برچسب
نشان داده می‌شود دیگر نیازی به استفاده از فضای خالی در ابتدایی یک گزینه نیست.
شاید بتوان این خصوصیت را یک برتری نسبت به روش پیشین در نظر گرفت.

\section{برچسب‌های دیگر ابزارها}

همانگونه که پیش از این بیان شده، برای ایجاد توانایی حمایت از مستندهای تولید شده
بر اساس دیگر استانداردها، دسته ای از برچسب‌ها به استاندارد \lr{Doxygen} اضافه
شده است. برای نمونه می‌توان به برچسب‌های \lr{arg} و \lr{li} اشاره کرد. این
برچسب‌ها در ابزارهای مستند سازی مانند \lr{QtDoc} و \lr{KDoc} مورد استفاده قرار
می‌گیرد.

بنابر این می‌توان از روش‌های معادل در ابزارهای دیگر فهرست‌های دلخاه را ایجاد
کرد. البته ایجاد این توانایی با اشکال‌هایی هم روبرو شده است که در ادامه به صورت
جداگانه به آنها پرداخته خواهد شد.

\subsection{\lr{QtDoc}}

هر فهرست در مستندهای \lr{QtDoc} با استفاده از برچسب \lr{list} و \lr{endlist}
ایجاد می‌شود. در حالت کلی یک فهرست به صورت زیر است:

\begin{latin}
\lstset{language=C++}  
\begin{lstlisting}[frame=single] 
/*! 
 * \list
 *	<list items>
 * \endlist
 */
\end{lstlisting}
\end{latin}

که در آن \lr{<list items>} می‌تواند یک گزینه و یا حتی یک فهرست دیگر باشد. بدهی
است که ایجاد فهرست‌های تودرتو با استفاده از این برچسب‌ها در \lr{QtDoc} ممکن
خواهد بود.

در این ابزار از برچسب \lr{li} برای ایجاد یک گزینه استفاده می‌شود. از این برچسب
نه تنها در ایجاد گزینه‌ها در یک فهرست بلکه برای ایجاد سطر در جدول نیز مورد
استفاده قرار می‌گیرد. در اینجا تنها کاربرد این برچسب در ایجاد یک فهرست مد نظر
خواهد بود.
در حالت کلی ایجاد یک گزینه در فهرست با استفاده از این برچسب به صورت زیر است:

\begin{latin}
 \lstset{language=C++}  
\begin{lstlisting}[frame=single] 
/*! 
 *  \li {item description}
 */
\end{lstlisting}
\end{latin}

پارامتر ورودی یک توصیف از گزینه مورد نظر است. متن هر گزینه با رسیدن به یکی از
موارد زیر پایان خواهد یافت:
\begin{itemize}
  \item یک سطر خالی
  \item بر چسب دیگری از \lr{li}
  \item برچسب پایان یک فهرست یا \lr{list}
  \item برچسب آغاز یک فهرست یا \lr{endlist}
\end{itemize}

\begin{warning}
گرچه می‌توان با استفاده از برچسب‌هایی مانند \lr{list} و \lr{endlist} فهرست‌های
تودرتو ایجاد کرد اما در \lr{Doxygen} این برچسب‌ها مورد حمایت قرار نگرفته است.
تنها برچسب مورد حمایت در این ابزار \lr{li} است. از این رو فهرست‌های ایجاد شده با
روش‌های معرفی شده در \lr{QtDoc} به صورت یک فهرست ساده و تک سطحی، در خروجی‌های
\lr{Doxygen} ظاهر خواهد شد.
\end{warning}

برای نمونه در مستند زیر یک فهرست ایجاد شده است که خرجی مشابه‌ای در \lr{Doxygen}
و \lr{QtDoc} دارد.

\begin{latin}
 \lstset{language=C++}  
\begin{lstlisting}[frame=single] 
/**
 * \list
 *  \li \c AlignLeft left alignment.
 *  \li \c AlignCenter center alignment.
 *  \li \c AlignRight right alignment
 *	  No other types of alignment are supported.
 * \endlist
 */
\end{lstlisting}
\end{latin}

\subsection{\lr{KDoc}}

این ابزار که معادل با ابزار \lr{JavaDoc} است، برای ایجاد مستند فنی بر اساس
برنامه‌های نوشته شده به زبان برنامه نویسی جاوا مورد استفاده قرار می‌گیرد. در این
ابزار با استفاده از برچسب \lr{arg} یک فهرست ایجاد می‌شود. این برچسب دارای تنها
یک پارامتر ورودی است. ساختار کلی این برچسب به صورت زیر است.

\begin{latin}
\lstset{language=C++}  
\begin{lstlisting}[frame=single] 
  /*! 
   *  \arg {item description}
   */
\end{lstlisting}
\end{latin}

پارامتر ورودی در حقیقت متنی است که گزینه را توصیف می‌کند. متن هر گزینه با قرار
دادن یک سطر خالی و یا شروع شدن یک برچسب جدید \lr{arg}  پایان می‌پذیرد. به این
نکته باید توجه کرد که با استفاده از این برچسب نمی‌توان فهرست‌های چند سطحی را
ایجاد کرد.

\begin{latin}
 \lstset{language=C++}  
\begin{lstlisting}[frame=single] 
/**
 * \arg \c AlignLeft left alignment.
 * \arg \c AlignCenter center alignment.
 * \arg \c AlignRight right alignment
 *	 No other types of alignment are supported.
 * 
 */
\end{lstlisting}
\end{latin}
