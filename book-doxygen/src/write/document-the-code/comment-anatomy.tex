% تعریف و سازماندهی مستند در این بخش ساختار کلی مستندها اورده می‌شود به بیان
% دیگر بیان می‌شود که متن مستند چگونه سازماندهی شده و نوشته می‌شود. چگونگی
% استفاده از برچسب‌ها و تعریف آنها و یا ایجاد پاراگراف و قسمت اشاره می‌شود.

\section{ساختار مستند}

پیش از این بیان شد که با استفاده از ساختارهای خاض دسته‌ای از \glspl{comment
block}ها به عنوان \glspl{document block} در نظر گرفته می‌شود. ابزارهای
\glspl{documenter} بر اساس \glspl{document block}های ایجاد شده مستند فنی کلی
سیستم را ساختاردهی و ایجاد می‌کنند. گرچه ساختاردهی و ایجاد مستند نهایی می‌تواند
به صورت خودکار ایجاد شود اما با ایجاد راهکارهای مناسب می‌توان امکاناتی جهت
مدیریت مستند فنی تولید شده توسط این ابزارها را بهبود بخشید. از این رو در بسیاری
از ابزارهای \glspl{documenter} نه تنها الگو و ساختارهایی برای \glspl{document
block} ارائه می‌شود بلکه دسته‌ای از دستورها و تنطیم‌ها برای مدیریت آنها نیز در
نظر گرفته می‌شود.

% FIXME: maso 1391: ساختار کلی بسته‌های مستند و رابطه آنها
% در این قسمت باید این ساختارها به صورت کامل تشریح شود. برای نمونه همواره پیش از
% هر موجودیت برنامه سازی مستند وارد می‌شود.

معمولا یک برنامه از بلاک های مختلفی تشکیل شده است. بطور مثال یک برنامه می تواند
شامل بلاک یک تابع، بلاک یک کلاس، بلاک یک متغیر و ... باشد. حال این توضیحات شامل
الحاق همه بلاک های توضیحاتی است که در یک تابع یا متد آورده شده است. بدین ترتیب
می توان تمام عبارت های وارد شده در کد که بصورت جدا از هم قرار دارند را به یکدیگر
متصل کرد. توصیف های مختصر معمولا بیان کننده توصیفات کلی یک بلاک از برنامه می
باشد. همچنین توصیف های مختصر مانند یک نام پیشنهادی، یک توصیف خلاصه،یک خط کوتاه
توضیح است که توصیف کننده کلی یک بخش از مستنداتی است که در آینده به کدها اضافه می
شود. توصیف های مختصر همچنین می توانند شامل مجموعه ای از توضیحات خلاصه یا توصیف
گر یک مجموعه از جزئیات پیاده سازی شده باشند. یادآور می شود که داشتن بیش از یک
توصیف خلاصه یا جزئیات مجاز می باشد(البته استفاده از این روش توصیه نمی شود).

\begin{note}
همانگونه که در بخش \ref{write/document-the-code/comment-block} گفته شد، به
روش‌های متفاوتی می‌توان یک \glspl{document block} را ایجاد کرد. در ادامه همواره
از یک روش یکتا برای ایجاد \glspl{document block} استفاده می‌شود که به صورت زیر
است:

\begin{Java}
/**
 * <Docuemnt block>
 */
\end{Java}

این به آن معنی نیست که دیگر ساختارهای مورد استفاده در ایجاد یک \glspl{document
block} معتبر نیست. از این رو می‌توان این روش را با هر روش دیگری جایگزین کرد. هدف
اصلی از این کار یک نواخت بودن ساختار کتاب است.
\end{note}

 مستندهای که در یک  \glspl{document block} وارد می شود را می‌توان بطور کلی به دو
دسته کلی تقسیم: توصیف کلی و توصیف جزئی. به طور معمول هر \glspl{document block}
با یک توصیف جزئی شروع شده و در ادامه به صورت مفصل به توصیف اجزای برنامه سازی
می‌پردازد. گرچه ایجاد این دو دسته مستند در \glspl{document block} کاملا اختیاری
است اما برخورد \glspl{documenter}ها با این مستندها متفاوت است. برای نمونه ابزار
\lr{JavaDoc} در گام نخست توصیف کوتاه از هر \glspl{document block} را تعیین کرده
و باقی مانده را به عنوان مستند کلی در نظر می‌گیرد در حالی که این روند در ابزاری
مانند \glspl{CDoc} کاملا برعکس است. این درحالی است که \glspl{doxygen} در این
مورد پویا بوده و بر اساس تنظیم‌های متفاوت می‌تواند روش‌های متفاوتی را به کار
گیرد. 

مستند جزئی عبارت است از یک توصیف کوتاه که به توصیف یک جز برنامه سازی می‌پردازد.
ابزارهای \glspl{documenter} متفاوت راهکارهای متفاوتی را برای تعیین یک مستند جزئی
دنبال می‌کنند. برای نمونه \lr{JavaDoc} پاراگراف نخست از هر \glspl{document
block} را به عنوان مستند جزئی در نظر می‌گیرد در حالت که در \lr{CDoc} و یا
\lr{QtDoc} از برچسب‌ها برای اینکار استفاده می‌شود.

برخلاف مستند جزئی که بسیار کوتاه و خلاصه است، مستند کلی به تشریح کامل یک برنامه
پرداخته و خود به بخش‌های متفاوتی تقسیم می‌شود. این مستند گاها شامل مثال بوده و
به قسمت‌های دیگر برنامه ارجاع دارد. بسته به میزان مستند سازی سیستم این قسمت
می‌تواند از بین چند خط تا چند بخش و زیر بخش متفاوت باشد. ساختار کلی مستند جزئی و
کلی در یک \glspl{document block} بر اساس استانداردهای \lr{JavaDoc} در زیر آورده
شده است.

\begin{Java}
/**
 * <Brief Document>
 *
 * <Detail Document>
 */
\end{Java}

ساختار مورد استفاده در \lr{JavaDoc} بسیار ساده است. در ساختارهای دیگر از
\glspl{doxygen:tag} برای تعیین مستند جزئی و کلی استفاده می‌شود که در ادامه به آن
پرداخته خواهد شد.

بسته ویژه مستند، یک بسته توضیحات در برنامه تویسی \lr{C/C++} با تعدادی علامت
گذاری اضافی است که \lr{doxygen} توسط آن ها تشخیص می دهد این قسمت از کد یک بخش از
مستندی است که در نهایت برای تولید گزارش از آن استفاده می شود.

برای مستند هایی که  خروجی آنها بصورت \lr{HTML} است، توصیف های خلاصه، بصورت
راهنمای ابزار \lr{(tooltip)}  مورد استفاده قرار می گیرد. در این روش قسمت های
مورد نظر از یک مستند، توضیحات آن قسمت از مستند وقتی که توسط موشواره اشاره می
شوند به نمایش در می آید.

% FIXME: maso 1391: تعریف برچسب باید آورده شود.
% برچسب برای تعیین یک دستور است. این کار به روش های متفاوت انجام می‌شود که باید در
% اینجا مورد بررسی قرار گیرد.

همانطور که بیان شد برای ارایه یک توضیح کلی برای یک بلاک از برنامه از توصیف خلاصه
استفاده می شود. برای پیاده سازی این گونه از توصیفات چندین روش وجود دارد که در
زیر بررسی می شود.

یکی از روش های توصیف خلاصه استفاده از فرمان \lr{\textbackslash{brief}} در ابتدای
یک بسته توضیح است. با استفاده از فرمان فوق می توان یک توصیف خلاصه از یک پاراگراف
را در ابتدای توضیحات آورد.

برای کلاس ها و پوشه ها، توصیف خلاصه در لیستی که در ابتدای مستندات یک صفحه آورده
می شود مورد استفاده قرار می گیرد. گستردگی یک توصیف خلاصه ممکن است از چند خط
فراتر رود. (البته تاکید می شود که توصیف خلاصه بیشتر از چند کلمه نباشد.)


\begin{C++}
/*! \brief Brief description.
 *         Brief description continued.
 *
 *  Detailed description starts here.
 */
\end{C++}

بطور معمول بعد از بیان یک خلاصه از جزئیات یک بلاک، موارد دیگری همچون نام
نویسنده، شماره ویرایش، تاریخ، خطاهای احتمالی، نسخه های پیشین، اخطارها و
استاندارد های پیاده سازی شده آورده می شود. البته موارد فوق زمانی مفید است که
توصیف یک کلاس یا پوشه مد نظر باشد وبرای موارد دیگر مانند توصیف متدها و توابع سعی
می شود از موارد دیگر بجز خلاصه استفاده نشود. مانند مثال زیر:

\begin{C++}
/*!
 * \brief Pretty nice class.
 *
 * This class is used to demonstrate a number of section commands.
 * 
 * \author Mostafa Barmshory
 * \version   4.1a
 * \date Aban 1391
 * \copyright GPL
 */
class SomeNiceClass {..};
\end{C++}

ابتدا باید یادآوری کرد که فایل پیکر بندی در مستند نویسی طبق استانداردهای
\lr{doxygen} فایلی است که در آن کلیه ساختار یک مستند و تنظیمات مربوط به مستند در
آن تعریف می شود.

اگر در فایل پیکر بندی تگ  \lr{JAVADOC-AUTOBRIEF} به \lr{yes} نشانده شود آنگاه
بسته های توضیح با قالب \lr{JavaDoc} بصورت پیش فرض و خودکار، خط اول از توضیحات را
تا رسیدن به اولین نقطه و پس از آن یک فضای خالی یا رفتن به سطر بعد را به عنوان
توصیف خلاصه در نظر می گیرد.
در توضیحات \lr{C++} نیز به همین ترتیب عمل می شود.به مثال های زیر دقت کنید.

\begin{C++}
/** Brief description which ends at this dot. Details follow
 *  here.
 */
\end{C++}

سومین مورد استفاده از توصیف های خلاصه درقالب \lr{C++} هنگامی است که یک
توضیح بیشتر از یک خط نیست. در این روش همه توصیف ها در یک خط می آید. مانند مثال
های زیر:

\begin{C++}
/// Brief description.
/** Detailed description. */

//! Brief description.

//! Detailed description
//! starts here.
\end{C++}

توجه کنید که خط خالی آمده در مثال آخر در جایی می آید که نیاز به جدا سازی توصیف
خلاصه از بسته شامل توصیف جزئیات باشد.
برای تنظیم چنین مواردی باید در فایل پیکر بندی باید تگ \lr{JAVADOC-AUTOBRIEF} به
\lr{no} نشانده شود.

همانطور که مشاهده شد در استفاده از توصیفات جزئیات \lr{doxygen} کاملا انعطاف پذیر
است. اگر چندین توصیف جزئیات وجود داشته باشد، همه آنها به هم الحاق خواهند شد.
اگر چندین توصیف جزئیات وجود داشته باشد که باید آنها را جدا از هم در مستند
بیاورید باید هر کدام از آنها را بصورت توضیحات خلاصه در خط های جداگانه قرار دهید.
مانند مثال زیر:

\begin{C++}
//! Brief description, which is
//! really a detailed description since it spans multiple lines.
/*! Another detailed description!
*/
\end{C++}

برخلاف بیشتر سیستم های مستند نویسی دیگر، در \lr{doxygen}  این امکان وجود دارد که
مستند مربوط به یک عضو در جلوی تعریف آن آورده شود. این روش مستند نویسی از این نظر
مفید است که می توان توضیحات را در فایل منبع بجای فایل سرآیند برای یک عضو از
برنامه آورد. این روش مستند نویسی باعث می شود که فایل سرآیند همیشه یکپارچگی خود
را حفظ کند.

گذاشتن مستندات بعد از عضوها:
توصیف های خلاصه بطور کلی بیان کننده یک دید کلی از یک کلاس ، فضای نام یا یک پوشه
هستند. بصورت پیش فرض فونت این قسمت بصورت ضخیم تر نمایش داده می شود(این توصیف می
تواند با نشاندن \lr{BRIEF-DESC} به \lr{no}درفایل پیکربندی پنهان شود.) هر دو مورد
توصیف های خلاصه و یا جزئیات برای قالب \lr{Qt} اختیاری هستند.

بصورت پیش فرض یک بسته مستند با قالب \lr{javaDoc} شبیه به یک قالب مستند \lr{Qt}
رفتار می کند.
این مورد طبق ویژگی های موجود در \lr{javaDoc} نیست،اما جمله اول از بسته مستند
بصورت خودکار مانند یک توصیف خلاصه رفتار می کند.
برای فعال کردن این خصوصیت باید
\lr{JAVADOC-AUTOBRIEF} را در فایل پیکربندی به \lr{yes} بنشانید. اگر شما این
خصوصیت را فعال کنید و بخواهید یک نقطه در بین یک جمله بدون پایان آن بگذارید،
باید یک اسلش و یک فاصله را بعد از ان قرار دهید. به مثال زیر نگاه کنید.

\begin{C++}
/** Brief description (e.g.\ using only a few words). Details follow. */
\end{C++}

بطور مشابه اگر بخواهید جمله اول از یک بسته مستند در قالب \lr{Qt} بصورت خودکار
مانند یک توصیف خلاصه رفتار کندباید \lr{QT-AUTOBRIEF} را درفایل پیکربندی
به\lr{yes} بنشانید.

\lr{doxygen} به شما اجازه می دهد تا مستندهای اعضا(شامل تابع های عمومی) در جلوی
تعریف آن \lr{definition} گذاشته شود. این روش در مستند سازی می تواند در فایل منبع
به منظور استفاده در فایل سرآیند قرار گیرد. این کار باعث می شود که فایل سرآیند
خاصیت یکپارچگی خود را حفظ کند و به مجری عضوها اجازه دسترسی مستقیم به مستند را
دهد. توصیف خلاصه قبل از اعلان عضو و توصیف جزئیات قبل از تعریف عضو قرار داده می
شود.



دستورهای ساخت یافته (مانند تمام دستورهای دیگر در مستند نویسی \lr{doxygen}) با یک
بک اسلش (\textbackslash) dh اتساین (@) شروع می شوند.
در ادامه این دستورات با یک نام دستور و یک یا چند پارامتر ادامه می یابد.

بطور مثال دستور استفاده شده جهت نشان دادن بسته مستند یک کلاس بصورت
\lr{\textbackslash class}
است. در زیر چند دستور که برای ساخت بسته مستند در مکان دیگر استفاده می شود آورده
شده است.

\begin{C++}
\struct to document a C-struct.
\union to document a union.
\enum to document an enumeration type.
\fn to document a function.
\var to document a variable or typedef or enum value.
\def to document a #define.
\typedef to document a type definition.
\file to document a file.
\namespace to document a namespace.
\package to document a Java package.
\interface to document an IDL interface.
\end{C++}

برخی از دستورهای یاد شده دارای یک یا چند آرگومان هستند. هر آرگومان یک محدوده خاص دارد.

\begin{itemize}
  \item     If <sharp> braces are used the argument is a single word.
  \item     If (round) braces are used the argument extends until the end of the line on which the command was found.
  \item     If {curly} braces are used the argument extends until the next paragraph. Paragraphs are delimited by a blank. line or by a section indicator.
\end{itemize}

البته باید یادآور شد که این دستورات بسیار کاملتر هستند.
در پیوند \cite{doxycommand} می توانید به دستورات کامل این مجموعه دست یابید.

دوباره یادآوری می شود برای شی های عمومی (تابع ها،توع های داده ای،توع شمارشی،
ماکروهاو...) مستند باید در همان فایلی که تعریف می شوند قرار گیرند.

\begin{C++}
/*! \file */

or a

/** @file */
\end{C++}

طبق مثال زیر می توان مستندهای یک فایل را بطور جداگانه از خود فایل پیاده سازی
کرد.همانطور که در مثال زیر مشخص است وقتی که مستند های یک فایل منبع  خارج از
خودفایل منبع تعریف می شوند امضای تمام توابعی که قرار است مستند آن ها جداگانه
آورده شود در فایل مستند قرار می گیرد. ضمنا برای مستند کردن هر قسمت از یک فایل
منبع که می تواند شامل کلاس، تابع، متغیر و... باشد دارای یک تگ مربوط  به خود است.
مثلا برا مستند کردن تابع از تگ \lr{/fn}، برای مستند کردن یک ماکرو ازتگ
\lr{/def} برای مستند کردن یک فایل از تگ \lr{/file} و برای مستند کردن یک متغیر
از تگ \lr{/var} استفاده می شود. به دلیل اینکه هر بسته توضیح شامل دستورات
ساختاری است، همه بسته های توضیح می توانند به جای دیگری یا یک فایل ورودی
انتقال یابند (مانند یک فایل حاوی منبع) بدون اینکه بر روی مستندات تولید شده تاثیر
داشته باشند. یکی از معایب استفاده از این روش این است که الگوها ممکن است جایگزین
یکدیگر شوند و یا بازنویسی شوند.

\begin{C++}
/*! \file structcmd.h
\brief A Documented file.

Details.
*/

/*! \def MAX(a,b)
\brief A macro that returns the maximum of \a a and \a b.

Details.
*/

/*! \var typedef unsigned int UINT32
\brief A type definition for a .

Details.
*/

/*! \var int errno
\brief Contains the last error code.

\warning Not thread safe!
*/

/*! \fn int open(const char *pathname,int flags)
\brief Opens a file descriptor.

\param pathname The name of the descriptor.
\param flags Opening flags.
*/

/*! \fn int close(int fd)
\brief Closes the file descriptor \a fd.
\param fd The descriptor to close.
*/

/*! \fn size_t write(int fd,const char *buf, size_t count)
\brief Writes \a count bytes from \a buf to the filedescriptor \a fd.
\param fd The descriptor to write to.
\param buf The data buffer to write.
\param count The number of bytes to write.
*/

/*! \fn int read(int fd,char *buf,size_t count)
\brief Read bytes from a file descriptor.
\param fd The descriptor to read from.
\param buf The buffer to read into.
\param count The number of bytes to read.
*/
#define MAX(a,b) (((a)>(b))?(a):(b))
typedef unsigned int UINT32;
int errno;
int open(const char *,int);
int close(int);
size_t write(int,const char *, size_t);
int read(int,char *,size_t);
\end{C++}


\subsection{تشریح یک بسته مستند}

در بخش قبل به بررسی نحوه وارد کردن مستند در کد با استفاده از استاندارد
\lr{doxygen} پرداختیم. همچنین در مورد اختلاف بین توصیف های خلاصه و جزییات  توضیح
داده شد. بعلاوه درباره استفاده از مستندهای ساخت یافته مطالبی بیان شد.

در این بخش به بررسی محتویات خود بسته مستند پرداخت می شود.
در \lr{Doxygen} از قالب های مختلف مستندات حمایت می شود.
در ساده ترین شکل می توان از متن ساده استفاده کرد. مورد یاد شده برای یک توصیف
کوتاه ایده آل می باشد.
برای بیان توصیف های بزرگتر اغلب به ساختارهای بیشتری نیاز پیدا خواهد شد. این
ساختارها می توانند شبیه یک بسته متن، یک لیست یا یک جدول ساده باشد.

\lr{doxygen} از قالب \lr{Markdown} ، که شامل بخش های توسعه یافته \lr{Markdown
Extra} می باشد، استفاده می کند.
\lr{Markdown} طوری طراحی شده است که برای خواندن و نوشتن بسیار آسان است.
\lr{Doxygen} همچنین خواندن بطور مستقیم از فایل های \lr{Markdown} را پشتیباینی می
کند.
برای کسب اطلاعات بیشتر به آدرس زیر مراجعه شود.

\href{http://www.stack.nl/~dimitri/doxygen/markdown.html}{markdown}
