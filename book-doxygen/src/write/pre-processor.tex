\chapter{پیش‌پردازنده}

\lr{Doxygen} 
یک پیش‌پردازنده برای زبان \lr{C} به صورت داخلی دارد. پرونده‌های منبعی که به
عنوان ورودی به \lr{doxygen} داده می‌شود را می‌توان با این پیش‌پردازنده تجزیه
کرد.

به صورت پیش‌فرض، \lr{doxygen} فقط پیش‌پردازش جزئی انجام می‌دهد. یعنی عبارات
ترجمه‌ای شرطی (مثل \lr{\#if}) و تعریف ماکروها را ارزیابی می‌کند، اما بسط ماکروها
را انجام نمی‌دهد. بنابراین اگر قطعه کد زیر را داشته باشید:
\begin{C++}
#define VERSION 200
#define CONST_STRING const char *

#if VERSION >= 200
  static CONST_STRING version = "2.xx";
#else
  static CONST_STRING version = "1.xx";
#endif
\end{C++}

آنگاه \lr{doxygen} به صورت پیش‌فرض کد زیر را به تجزیه‌کننده‌اش تحویل می‌دهد.
\begin{C++}
#define VERSION
#define CONST_STRING
static CONST_STRING version = "2.xx";
\end{C++}

با مقداردهی تگ \lr{ENABLE\_PREPROCESSING} با مقدار \lr{NO} در پرونده پیکربندی،
می‌توانید تمام پیش‌پردازش‌ها را غیرفعال کنید. در این صورت در مورد قطعه کد بالا،
\lr{doxygen} هر دو عبارت را می‌خواند. یعنی:
\begin{C++}
static CONST_STRING version = "2.xx";
static CONST_STRING version = "1.xx";
\end{C++}

در صورتی که بخواهید ماکروی \lr{CONST\_STRING} در قطعه کد بالا را بسط دهید، باید تگ \lr{MACRO\_EXPANSION} را در 
پرونده پیکربندی با مقدار \lr{YES} مقداردهی کنید. در این صورت نتیجه پس از پیش پردازش به صورت زیر خواهد بود:
\begin{C++}
#define VERSION
#define CONST_STRING
static const char * version = "1.xx";
\end{C++}
توجه کنید که \lr{doxygen} در این حالت تمام تعاریف ماکروها را (در صورت نیاز به صورت  بازگشتی) بسط خواهد داد. 
این کار در اغلب موارد بسیار سنگین است. بنابراین \lr{doxygen}  همچنین به شما اجازه می‌دهد تا فقط آن تعاریفی که 
صراحتاً تعیین می‌کنید را بسط دهید. برای این کار باید تگ \lr{EXPAND\_ONLY\_PREDEF} را با \lr{YES} مقداردهی کنید و 
تعاریف ماکروهای مورد نظر خود را بعد از تگ \lr{PREDEFINED} یا تگ \lr{EXPAND\_AS\_DEFINED} مشخص کنید. 
یک مثال خاص که به کمک گرفتن از پیش‌پردازنده نیاز دارد، مربوط به توسعه زبانی \lr{\_\_declspec} مایکروسافت است. 
در زیر یک تابع مثال آمده است:
\begin{latin}
\lstset{language=C++}  
\begin{lstlisting}[frame=single]
extern "C" void __declspec(dllexport) ErrorMsg( String aMessage,...);
\end{lstlisting}
\end{latin}
وقتی کاری انجام نشود، \lr{doxygen} دچار سردرگمی شده و \lr{\_\_declspec} را به عنوان یک تابع در نظر می‌گیرد. 
برای کمک به \lr{doxygen}، می‌توان تنظیمات زیر را برای پیش‌پردازنده به کار برد:
\begin{latin}
\lstset{language=C++}  
\begin{lstlisting}[frame=single] 
ENABLE_PREPROCESSING    = YES
MACRO_EXPANSION         = YES
EXPAND_ONLY_PREDEF      = YES
PREDEFINED              = __declspec(x)=
\end{lstlisting}
\end{latin}
این کار اطمینان می‌دهد که قبل از اینکه کد منبع توسط \lr{doxygen} تجزیه شود، \lr{\_\_declspec(dllexport)} حذف می‌شود.

به عنوان یک مثال پیچیده‌تر فرض کنید قطعه کد گیج‌کننده زیر را از یک کلاس پایه انتزاعی به نام \lr{IUnknown} دارید.
\begin{latin}
\lstset{language=C++}  
\begin{lstlisting}[frame=single] 
/*! A reference to an IID */
#ifdef __cplusplus
#define REFIID const IID &
#else
#define REFIID const IID *
#endif


/*! The IUnknown interface */
DECLARE_INTERFACE(IUnknown)
{
  STDMETHOD(HRESULT,QueryInterface) (THIS_ REFIID iid, void **ppv) PURE;
  STDMETHOD(ULONG,AddRef) (THIS) PURE;
  STDMETHOD(ULONG,Release) (THIS) PURE;
};
\end{lstlisting}
\end{latin}
بدون بسط ماکرو، \lr{doxygen} دچار سردرگمی می‌شود، اما ممکن است نخواهیم ماکروی \lr{REFIID} بسط داده شود، 
چون مستند شده است و  کاربری که مستند را می‌خواند باید هنگام پیاده‌سازی واسط از آن استفاده کند. با قرار دادن تنظیمات
زیر در پرونده پیکربندی: 
\begin{latin}
\lstset{language=C++}  
\begin{lstlisting}[frame=single] 
ENABLE_PREPROCESSING = YES
MACRO_EXPANSION      = YES
EXPAND_ONLY_PREDEF   = YES
PREDEFINED           = "DECLARE\_INTERFACE(name)=class name" \
                       "STDMETHOD(result,name)=virtual result name" \
                       "PURE= = 0" \
                       THIS_= \
                       THIS= \
        __cplusplus
\end{lstlisting}
\end{latin}
می‌توانیم مطمئن باشیم که نتیجه مناسب زیر به تجزیه‌کننده \lr{doxygen} تحویل داده می‌شود:
\begin{latin}
\lstset{language=C++}  
\begin{lstlisting}[frame=single] 
/*! A reference to an IID */
#define REFIID

/*! The IUnknown interface */
class  IUnknown
{
  virtual  HRESULT   QueryInterface ( REFIID iid, void **ppv) = 0;
  virtual  ULONG   AddRef () = 0;
  virtual  ULONG   Release () = 0;
};
\end{lstlisting}
\end{latin}
توجه کنید که تگ \lr{PREDEFINED}، تعاریف ماکروهای شبیه تابع (مثل \lr{DECLARE\_INTERFACE})، جایگزینی‌های ماکروهای 
معمولی مثل (مثل \lr{PURE} و \lr{THIS}) و تعاریف ساده (مثل \lr{\_\_cplusplus}) را می‌پذیرد. 

همچنین توجه کنید که آن تعاریف پیش‌پردازنده که معمولاً به طور خودکار به وسیله پیش‌پردازنده تعریف شده‌اند (مثل \lr{\_\_cplusplus})، 
باید به صورت دستی به تجزیه‌کننده \lr{doxygen} معرفی شوند (این کار به این دلیل است که این تعاریف اغلب مخصوص 
بستر\footnote{\lr{Platform}} یا مترجم\footnote{\lr{Compiler}} هستند).

%در بعضی موارد شما  ممکن است بخواهید یک نام ماکرو  یا تابع را با چیز دیگری جایگزین کنید بدون اینکه 
%این جایگزینی گزارش شود. شما می‌توانید این کار را با استفاده از عملگر =: به جای = انجام دهید.
%به عنوان مثال فرض کنید قطعه کد زیر را داریم:
%\begin{latin}
%\lstset{language=C++}  
%\begin{lstlisting}[frame=single]
%
%\end{lstlisting}
%\end{latin}

%بنابراین تنها راهی که این قطعه، به عنوان یک تعریف کلاس برای کلاس \lr{qlist} در \lr{doxygen} تفسیر شود به این صورت می باشد:
%\begin{latin}
%\lstset{language=C++}  
%\begin{lstlisting}[frame=single] 
%PREDEFINED = QListT:=QList
%\end{lstlisting}
%\end{latin}
%
%در اینجا یک مثال از \lr{Valter Minute} و \lr{Reyes Ponce} آورده شده که به \lr{doxygen} کمک میکند 
%که کد را  در کتابخانه‌های \lr{ATL} و \lr{MFC} مایکروسافت شناور سازد.
%\begin{latin}
%\lstset{language=C++}  
%\begin{lstlisting}[frame=single] 
%PREDEFINED           = "DECLARE\_INTERFACE(name)=class name" \
%                       "STDMETHOD(result,name)=virtual result name" \
%                      "PURE= = 0" \
%                       THIS\_= \
%                       THIS= \
%                       DECLARE\_REGISTRY\_RESOURCEID=// \
%                       DECLARE\_PROTECT\_FINAL\_CONSTRUCT=// \
%                       "DECLARE\_AGGREGATABLE(Class)= " \
%                      "DECLARE\_REGISTRY\_RESOURCEID(Id)= " \
%                       DECLARE\_MESSAGE\_MAP= \
%                       BEGIN\_MESSAGE\_MAP=/* \
%                       END\_MESSAGE\_MAP=*/// \
%                       BEGIN\_COM\_MAP=/* \
%                       END\_COM\_MAP=*/// \
%                       BEGIN\_PROP\_MAP=/* \
%                       END\_PROP\_MAP=*/// \
%                       BEGIN\_MSG\_MAP=/* \
%                       END\_MSG\_MAP=*/// \
%                       BEGIN\_PROPERTY\_MAP=/* \
%                       END\_PROPERTY\_MAP=*/// \
%                       BEGIN\_OBJECT\_MAP=/* \
%                      END\_OBJECT\_MAP()=*/// \
%                       DECLARE\_VIEW\_STATUS=// \
%                       "STDMETHOD(a)=HRESULT a" \
%                       "ATL\_NO\_VTABLE= " \
%                       "\_\_declspec(a)= " \
%                       BEGIN\_CONNECTION\_POINT\_MAP=/* \
%                       END\_CONNECTION\_POINT\_MAP=*/// \
%                       "DECLARE\_DYNAMIC(class)= " \
%                       "IMPLEMENT\_DYNAMIC(class1, class2)= " \
%                       "DECLARE\_DYNCREATE(class)= " \
%                       "IMPLEMENT\_DYNCREATE(class1, class2)= " \
%                       "IMPLEMENT\_SERIAL(class1, class2, class3)= " \
%                       "DECLARE\_MESSAGE\_MAP()= " \
%                       TRY=try \
%                       "CATCH\_ALL(e)= catch(...)" \
%                       END\_CATCH\_ALL= \
%                       "THROW\_LAST()= throw"\
%                       "RUNTIME\_CLASS(class)=class" \
%                       "MAKEINTRESOURCE(nId)=nId" \
%                       "IMPLEMENT\_REGISTER(v, w, x, y, z)= " \
%                       "ASSERT(x)=assert(x)" \
%                       "ASSERT\_VALID(x)=assert(x)" \
%                       "TRACE0(x)=printf(x)" \
%                       "OS\_ERR(A,B)={ #A, B }" \
%                       \_\_cplusplus \
%                       "DECLARE\_OLECREATE(class)= " \
%                       "BEGIN\_DISPATCH\_MAP(class1, class2)= " \
%                       "BEGIN\_INTERFACE\_MAP(class1, class2)= " \
%                       "INTERFACE\_PART(class, id, name)= " \
%                       "END\_INTERFACE\_MAP()=" \
%                       "DISP\_FUNCTION(class, name, function, result, id)=" \
%                       "END\_DISPATCH\_MAP()=" \
%                       "IMPLEMENT\_OLECREATE2(class, name, id1, id2, id3, id4,\
%                        id5, id6, id7, id8, id9, id10, id11)="
%\end{lstlisting}
%\end{latin}
همانطور که مشاهده می‌شود پیش‌پردازنده \lr{doxygen} بسیار قدرتمند است، با این حال اگر نیاز به انعطاف‌پذیری بیشتری 
دارید همواره می‌توانید یک پالایه\footnote{\lr{Filter}} ورودی بنویسید و آن را بعد از تگ \lr{INPUT\_FILTER} قرار دهید.

اگر مطمئن نیستید که عملیات پیش‌پردازش \lr{doxygen} چه اثری خواهد داشت،  می‌توانید \lr{doxygen} را به صورت زیر 
از طریق خط فرمان اجرا کنید:
\begin{latin}
\lstset{language=C++}  
\begin{lstlisting}[frame=single] 
doxygen -d Preprocessor
\end{lstlisting}
\end{latin}
این فرمان به \lr{doxygen} دستور خواهد داد که منابع ورودی را بعد از اینکه عملیات پیش‌پردازش انجام شد، در 
خروجی استاندارد چاپ کند. به عبارتی با این فرمان شما علاوه بر اجرای \lr{doxygen} برای تولید مستند می‌توانید در خروجی 
خط فرمان محتویات پرونده‌های منبع را پس از پیش‌پردازش انجام شده به وسیله \lr{doxygen} ببینید. (نکته: برای غیرفعال 
کردن پیغام‌ها و خروجی‌های دیگر، تگ‌های \lr{QUIET} و \lr{WARNING} را در پرونده پیکربندی 
به صورت \lr{QUIET=YES} و \lr{WARNING=NO} مقداردهی کنید.)
