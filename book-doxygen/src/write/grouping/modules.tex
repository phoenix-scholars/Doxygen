%
% حق نشر 1390-1402 دانش پژوهان ققنوس
% حقوق این اثر محفوظ است.
% 
% استفاده مجدد از متن و یا نتایج این اثر در هر شکل غیر قانونی است مگر اینکه متن حق
% نشر بالا در ابتدای تمامی مستندهای و یا برنامه‌های به دست آمده از این اثر
% بازنویسی شود. این کار باید برای تمامی مستندها، متنهای تبلیغاتی برنامه‌های
% کاربردی و سایر مواردی که از این اثر به دست می‌آید مندرج شده و در قسمت تقدیر از
% صاحب این اثر نام برده شود.
% 
% نام گروه دانش پژوهان ققنوس ممکن است در محصولات دست آمده شده از این اثر درج
% نشود که در این حالت با مطالبی که در بالا اورده شده در تضاد نیست. برای اطلاع
% بیشتر در مورد حق نشر آدرس زیر مراجعه کنید:
% 
% http://dpq.co.ir/licence
%
\section{\glspl{doxygen:module}‌}

\glspl{doxygen:module} بندی روشی است که با استفاده از آن می‌توان مستندهای متفاوتی را در یک
مستند مشترک گرد آوری کرد. یک پیمانه را می‌توان به صورت یک گردایه از مستندهای مستقل
تصور کرد که به صورت یک پارچه در یک مکان گردآوری شده است. مستند
پروده‌های دیگر، کلاسها، روندها، متغیرها، تعریف‌ها و یا حتی دسته‌های دیگر  می‌توانند به عنوان
بخشی از یک \glspl{doxygen:module} باشند.
 
توانایی اضافه کردن یک \glspl{doxygen:module}، در \glspl{doxygen:module}‌ای دیگر قابلیت ایجاد ساختار سلسله مراتبی در
دسته‌بندی را فراهم می‌کند. ساختار سلسله مراتبی نه تنها مدیریت مستندهای بزرگ
را ممکن می‌سازد بلکه مستند را در دسترس‌تر خواهد کرد به گونه‌ای که کاربران
می‌توانند به سادگی به مستندهای مورد نیاز خود دست یابند. این دسته بندی می‌تواند بر اساس خواص
مشترک ایجاد شود اما به طور معمول این دسته‌بندی محتوایی است و تمام مستندهای مرتبط را باهم 
دسته بندی کرده و در یک ساختار سلسله مراتبی سازماندهی می‌کند.

\subsection{ایجاد \glspl{doxygen:module}}

برای تعریف یک \glspl{doxygen:module} از برچسب \lr{defgroup} استفاده می شود. با قرار دادن این برچسب
در یک بسته مخصوص (بسته‌های مخصوص را ببینید) یک دسته ایجاد خواهد شده. قالب کلی
استفاده از این برچسب به صورت زیر است:
\begin{C++}
/**
 * \defgroup <name> (group title)
 */
\end{C++}

نخستین \glspl{argument} این برچسب، یک شناسه را برای گروه تعیین می‌کند که باید در تمام
مستند به صورت یکتا تعریف شده باشد. این شناسه (همانند تمام شناسه‌های دیگر) می‌بایست
تنها از یک واژه تشکیل شده باشد. \glspl{argument} دوم عنوان برای دسته
تعیین می‌کند، که در مستندها برای نمایش دسته مورد استفاده قرار می گیرد. 
به عنوان نمونه، کد زیر یک دسته به نام \lr{Example}  را ایجاد می‌کند:

\begin{C++}
/*
 * \defgroup example_group Example
 * This is example group
 */
\end{C++}

مستند مربوط به هر دسته درست بعد از تعریف آن قرار نوشته می‌شود که یک شرح کامل از
آن خواهد بود. این مستند می‌تواند
شامل بخش بندی خاص بوده و از تمام امکانات موجود در مستند سازی بهره ببرد.

\subsection{اضافه کردن به دسته}
اضافه کردن مستند به یک دسته خواص با استفاده از برچسب \lr{ingroup} انجام می‌شود. با
استفاده از این برچسب می‌توان به سادگی یک کلاس، پرونده، متغیر، و یا غیره را به یک یا چند
دسته خواص اضافه کرد. ساختار کلی این برچسب به صورت زیر است.

\begin{C++}
\ingroup (<groupname> [<groupname><groupname>])
\end{C++}
تنها نوع \glspl{argument} در این برچسب، شناسه \glspl{doxygen:module}‌ای است که مستند به آن 
تعلق دارد.

\begin{warning}
امکان اضافه کردن یک مستند به چندین دسته، در نسخه‌های نخستین این بسته فراهم نبوده، از
این رو همواره تلاش کنید که مستندها را به گونه‌ای دسته بندی کنید که هر بخش در یک \glspl{doxygen:module}
قرار گیرد.
\end{warning}

گرچه با استفاده از این برچسب می‌توان یک مستند را به هر دسته‌ای اضافه کرد اما در
بسیاری از حالت‌ها استفاده از آن منجر به به زیاد شدن برچسب‌ها و مشکل شدن مستند سازی
در پروژه‌ها خواهد شد. حالتی را فرض کنید که در آن تعداد زیادی متغیر و یا بسته‌های 
مستند کوتاه و پشت سرهم نوشته شده‌اند و هدف قراردادن تمام این بسته‌های مستند در یک
\glspl{doxygen:module} خاص است. با استفاده از این برچسب باید به ناچار هرکدام از بسته‌ها را به صورت
جداگانه برچسب‌گذاری کرد. 

زمانی که مستند‌ها همگی در یک پرونده و به صورت پشت سرهم ایجاد شده باشند، می‌توان با
استفاده از برچسب \lr{ \ \{  \ \}} به صورت فیزیکی مستند‌ها را در یک گروه قرار داد.
این روش نه تنها در ایجاد دسته‌های و گروه‌ها بلکه در بسیاری از موارد دیگر
مورد استفاده قرار می‌گیرد. راهکار اصلی در این روش استفاده از یک برچسب شروع و پایان
برای تمام اضای یک پیمانه است. در زیر یک نمونه استفاده از این عبارت آمده است.

\begin{C++}
/**
 * \defgroup gexample Example
 * \{
 */
 ...
/**\}*/
\end{C++}
به بیان دیگر برچسب  \lr{ \ \{  \ \}} یک محدوده ایجاد می‌کند که تمام مستندهای تعریف شده در آن
در \glspl{doxygen:module} متناظر با آن محدوده قرار می‌گیرد. اما نکته‌ای که باید در نظر داشت این است که
در اینجا تعریف یک \glspl{doxygen:module} و اضافه کردن مستندها با هم انجام می‌شود.

این روش را می‌توان در ایجاد \glspl{doxygen:module}‌های تودرتو نیز به کاربرد. تنها کافی است که دسته
جدید در محدوده یک دسته دیگر تعریف شود، به این ترتیب دسته دوم به عنوان یک زیر
دسته از دسته اول در نظر گرفته خواهد شد.

زمانی که از یک شناسه مشترک برای تعریف دو دسته استفاده شود، در فرآیند ایجاد
مستند خطا صادر خواهد شده و در بسیاری از موارد این فرآیند با مشکل اساسی روبرو
می‌شود (به بیان دیگر استفاده از یک شناسه برای تعریف دو دسته
مجزا، مجاز نمی باشد). یک روش جایگزین برای تعریف دسته استفاده از برچسب
\lr{addtogroup} است. قالب کلی این برچسب مانند برچسب تعریف یک گروه است. تعریف کلی
این برچسب به صورت زیر است.
\begin{C++}
\addtogroup <name> [(title)]
\end{C++}

یک تفاوت مهم این برچسب با برچسب تعریف دسته  اختیاری بودن عنوان برای \glspl{doxygen:module}
در این برچسب است. اگر یک \glspl{doxygen:module} با یک شناسه تعریف شده باشد، استفاده از این 
دستور برای ایجاد و اضافه کردن یک بسته مستند به گروه مورد نظر با خطا روبرو نخواهد شد.
ام زمانی که هیچ \glspl{doxygen:module}‌ای شناسه مورد نظر یافت نشود، یک \glspl{doxygen:module} ایجاد شده 
و مستندهای مورد نظر به آن اضافه می‌شود. از این رو می‌توان گفت که این تکنیک فرایند تولید
مستند را انعطاف پذیر خواهد کرد.

همان گونه که پیش از این اشاره شده، با ستفاده از برچسب‌های \lr{ \ \{ \ \}} یک محدوده
و مستندهای مورد نظر در آن تعریف می‌شود. تمام مستد‌های تعریف شده در این محدوده به \glspl{doxygen:module} 
مربوط به آن اضافه خواهد شد. این روش را نیز می‌توان با برچسب \lr{addtogroup} به کار برد.
با استفاده از این روش می‌توان مستندها را به صورت فیزیکی دسته بندی کرده و آنها را در
هر گروه دلخواه قرار داد (در تکه مستند زیر استفاده از این روش نشان داده شده است).

\begin{C++}
/**
 * \addtogroup gexample
 * \{
 */
...
/**
 * \}
 */
\end{C++}

با استفاده از برچسب \lr{ref} می‌توان یک پیونده به یک \glspl{doxygen:module} ایجاد کرد. این برچسب که در بخش‌های
بعد مورد بررسی قرار می‌گیرد از یک \glspl{parameter} به عنوان شناسه استفاده می‌کند که مستند مورد نظر
کاربر را تعیین می‌کند. از این برچسب برای ایجاد پیوند به یک \glspl{doxygen:module} نیز استفاده می شود
که \glspl{parameter} ورودی آن شناسه \glspl{doxygen:module} است.

\begin{note}
همواره باید این نکته را به یاد داشت که موجودیت‌های مانند پروند، کلاس ویا فضای
نام را می‌توان در دسته‌های متفاوتی قرار داد در حالی که موجودیت‌های مانند
متغیر، و متد را نمی‌تواند به بیش از یک دسته اضافه کرد. دلیل اصلی این کار پیش‌گیری از
مبهم شدن مستندها است.

با این فرض زمانی که یک موجودیت (به صورت غیرمجاز) به دو یا چند دسته اضافه شود، چه
روی خواهد داد؟ تمام برچسب‌هایی که یک مستند را در یک دسته جای می‌دهند دارای یک
اولویت هستند، از این رو مستند به دسته ای اضافه می‌شود که با استفاده از برچسب با
اولیت بیشتر تعیین شده است. اولویت میان برچسب‌ها عبارت اند از: \lr{ingroup}
\lr{defgroup} \lr{addtogroup} \lr{weakgroup} .
برچسب \lr{weakgroup} همانند برچسب \lr{addtogroup} است با این تفاوت که اولویت آن
نسبت به \lr{addtogroup} کمتر است.

\end{note}

