

\cite <label>

Adds a bibliographic reference in the text and in the list of bibliographic
references. The <label> must be a valid BibTeX label that can be found in one of
the .bib files listed in CITE_BIB_FILES. For the LaTeX output the formatting of
the reference in the text can be configured with LATEX_BIB_STYLE. For other
output formats a fixed representation is used. Note that using this command
requires the bibtex tool to be present in the search path.

CITE_BIB_FILES

The CITE_BIB_FILES tag can be used to specify one or more bib files
containing the reference definitions. This must be a list of .bib files. The
.bib extension is automatically appended if omitted. This requires the
bibtex tool to be installed. See also http://en.wikipedia.org/wiki/BibTeX
for more info. For LaTeX the style of the bibliography can be controlled
using LATEX_BIB_STYLE. See also \cite for info how to create references.


LATEX_BIB_STYLE

The LATEX_BIB_STYLE tag can be used to specify the style to use for the
bibliography, e.g. plainnat, or ieeetr. The default style is plain. See
http://en.wikipedia.org/wiki/BibTeX and \cite for more info.

Some bibliography styles

The PDF file bibstyles.pdf illustrates how these bibliographic styles render
citations and reference entries:

1: ieeetr 2: unsrt 3: IEEE 4: ama 5: cj 6: nar 7: nature 8: phjcp 9: is-unsrt
10: plain 11: abbrv 12: acm 13: siam 14: jbact 15: amsplain 16: finplain 17:
IEEEannot 18: is-abbrv 19: is-plain 20: annotation 21: plainyr 22: decsci 23:
jtbnew 24: neuron 25: cell 26: jas99 27: abbrvnat 28: ametsoc 29: apalike 30:
jqt1999 31: plainnat 32: jtb 33: humanbio 34: these 35: chicagoa 36: development
37: unsrtnat 38: amsalpha 39: alpha 40: annotate 41: is-alpha 42: wmaainf 43:
alphanum 44: apasoft

(The web page http://www.cs.stir.ac.uk/~kjt/software/latex/showbst.html also
illustrates several bibliography styles for easy comparison.) Of the 44 styles
listed above, the first 21 insert just a number in brackets at the point of
citation [2], while #22-37 use some variation of author/year [K.S.Narenda and
K.Parthsarathy, 1990], and the rest use some idiosyncratic reference code. Some
of the styles re-order the references in the bibliography in alphabetical order
of author, while others list them in the order that they are first cited.

Several of these styles are part of all LaTeX installations, and others can be
downloaded from http://www.tug.org/tex-archive/biblio/bibtex/contrib/. Each is a
file with the suffix .bst; for example, to use abbrvnat style, you must have the
file abbrvnat.bst installed in your LaTeX directory, or in your current working
directory, or anywhere where LaTeX can find it.

Those dissatisfied with the 100+ styles available online can design their own,
if they wish; see Oren Patashnik's Designing BibTeX Styles.
 