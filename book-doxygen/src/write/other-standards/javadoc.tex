% در این مستند به ابزارها و روشهای نوشتن مستند پرداخته می‌شود که از مستند سازی java
% وارد Doxygen شده است.
 
\section{تولیدگر مستند جاوا}
امروزه ابزارها و روش‌های متفاوتی برای تولید خودکار مستندهای توسعه به وجود آمده‌اند. یکی از پرکاربرد ترین ابزار تولید مستند توسعه، \lr{Java Doc} است\cite{javadocsun}. مهمترین ابزار مورد استفاده در ایجاد مستند هسته جاوا و بسیاری از بسته‌های 
نرم‌افزاری همین ابزار است. کاربرد گسترده این ابزار تا جایی است که در بسیاری از موارد تنها با استفاده از استاندارهای تعریف شده
در آن مستند بسته‌های نرم‌افزاری توسعه یافته است.

کاربرد این ابزار تا جایی پیش رفته است که نه تنها در تولید مستند از کد منبع، بلکه در تولید مستند بر اساس مدل‌های ایجاد شده
نیز به کار برده می‌شود. در بسیاری از موارد توسعه مدل هم گام با توسعه نرم‌افزار پیش می‌رود، از این رو می‌توان مدل را نیز در 
قالب کد منبع توسعه داد. امروزه روش‌ها و قراردادهای متفاوتی در زمینه ایجاد و تعریف مدل در متن برنامه و ایجاد مستند بر
اساس ابزار تولید مستند \lr{JavaDoc} توسعه یافته است\cite{Kramer:1999:ADS:318372.318577}.

نه تنها سادگی در به کارگیری این ابزار، بلکه توانایی‌های این نرم افزار در برخورد با چالش‌های موجود در تولید مستند توسعه، منجر به استفاده گسترده از این ابزار در توسعه مستند‌ها شده است\cite{Leslie:2002:UJX:584955.584971}. کاربردهای متفاوت و
سادگی به کارگیری این ابزار، آن را به عنوان یک ابزار پایه در برنامه سازی با جاوا تبدیل کرده به گونه‌ای که در کتاب‌های برنامه سازی
و درسی جاوا همواره به آن پرداخته می‌شود\cite{Schildt:2000:JCR:557816}.

برچسب‌ها در \lr{Java Doc} با استفاده از نشانه \lr{@} مشخص می‌شود در حالی که قرارداد \lr{Doxygen} 
استفاده از علامت اسلش\footnote{Slash} است. برای نمونه، در \lr{Java Doc} خروجی یک متد به صورت زیر تعیین
می‌شود.
\begin{latin}
\lstset{language=C++}  
\begin{lstlisting}[frame=single]
  /**
   * @return <document>
   */
\end{lstlisting}
\end{latin}
گرچه قرارداد پیش‌فرض در \lr{Doxygen} استفاده از نشانه اسلش است اما،  امکان تعیین برچسب‌ها به روش
مورد استفاده در \lr{Java Doc} نیز فراهم شده است. در \lr{Doxygen} تمام برچسب‌های استاندارد
موجود در \lr{Java Doc} قابل استفاده است، از این رو می‌توان با استفاده از این ابزار مستند توسعه را برای کدهایی ایجاد
کرد که بر اساس قرارداد موجود در \lr{Java Doc} مستند شده‌اند.

\lr{Java Doc}، در فرایند تولید مستند توسعه، برچسب‌هایی تعریف نشده را ندیده می‌گیرد و آنها را در خروجی وارد نمی‌کند از این
رو  نوشتن برچسب‌ها بر اساس قراردادهای \lr{Java Doc} یک مزیت به شمار می‌آید. مستند‌های که به این روش نوشته می‌شوند نه
تنها می‌توانند در \lr{Java Doc} به کار گرفته شوند بلکه در \lr{Doxygen} نیز قابل استفاده هستند.