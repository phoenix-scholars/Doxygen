%
% حق نشر 1390-1402 دانش پژوهان ققنوس
% حقوق این اثر محفوظ است.
% 
% استفاده مجدد از متن و یا نتایج این اثر در هر شکل غیر قانونی است مگر اینکه متن حق
% نشر بالا در ابتدای تمامی مستندهای و یا برنامه‌های به دست آمده از این اثر
% بازنویسی شود. این کار باید برای تمامی مستندها، متنهای تبلیغاتی برنامه‌های
% کاربردی و سایر مواردی که از این اثر به دست می‌آید مندرج شده و در قسمت تقدیر از
% صاحب این اثر نام برده شود.
% 
% نام گروه دانش پژوهان ققنوس ممکن است در محصولات دست آمده شده از این اثر درج
% نشود که در این حالت با مطالبی که در بالا اورده شده در تضاد نیست. برای اطلاع
% بیشتر در مورد حق نشر آدرس زیر مراجعه کنید:
% 
% http://dpq.co.ir/licence
%
% در این مستند به ابزارها و روشهای نوشتن مستند پرداخته می‌شود که از مستند سازی Qt
% وارد Doxygen شده است.
\section{مستندگر کیوتی}

\lr{QtDoc} یا \lr{QDoc} ابزاری جهت ایجاد مستند توسعه بر اساس متن برنامه است که توسط توسعه دهندگان
بسته نرم‌افزاری \lr{Qt} مورد استفاده قرار می‌گیرد. این ابزار بر اساس مستند‌های نوشته شده در 
پرونده‌های \lr{.cpp} و \lr{.qdoc}،  مستند توسعه را در قالب‌هایی مانند \lr{XML} و یا دیگر قالب‌های
متنی ایجاد می‌کند. توجه به این نکته مهم است که، در این نرم‌افزار پرونده‌های سرایند\footnote{\lr{Header Files}}
در فرآیند استخراج متن نادیده گرفته می‌شوند. متن‌های مستند در \lr{QDoc} با استفاده از نشانه تعجب 
در متن برنامه مشخص می‌شود. در نمونه زیر ساختار نوشتن یک مستند بر اساس استاندارد \lr{QDoc}
نشان داده شده است.
\begin{latin}
 \lstset{language=C++}  
\begin{lstlisting}[frame=single] 
  /*! 
      \class QObject
      \brife QObject is base class in Qt.
      
      QObject is heart of the Qt lib.
   */
\end{lstlisting}
\end{latin}

همان گونه که در نمونه بالا قابل مشاهده است، برچسب‌ها در استاندارد \lr{QDoc} همانند 
برچسب‌ها در \lr{Doxygen}  با استفاده از \textbackslash مشخص می‌شوند. تمام برچسب‌های
موجود در استاندارد مستند نویسی \lr{QDoc} را می‌توان به سه دسته تقسیم کرد، که عبارت‌اند از:
\begin{itemize}
 \item موضوع
 \item متن
 \item نشانک‌ها
\end{itemize}
تمام برچسب‌های تعریف شده در استاندارد \lr{QDoc} ، در \lr{Doxygen} نیز تعریف شده است، از این رو 
با استفاده از \lr{Doxygen} می‌توان مستند توسعه را بر اساس متن برنامه‌های ایجاد کرد که بر اساس
استانداردهای \lr{QDoc}‌ایجاد شده اند.در این بخش به صورت گذرا، هرکدام از این دسته‌ها را بررسی 
خوا‌هیم کرد.
\subsection{موضوع}
برچسب‌های موضوع تعیین می‌کنند که یک مستند نوشته شده به کدام قسمت از متن برنامه 
مربوط می‌شود. دسته‌ای از این برچسب‌ها نیز برای ایجاد صفحه‌هایی از مستند‌ها و دسته بندی
کردن آنها به کار می‌روند. زمانی که با استفاده از برچسب‌های موضوع تعیین نمی‌شود که مستند
به کدام قسمت از متن مربوط می‌شود، مستند به کدی نسبت داده می‌شود که بلافاصله بعد از 
مستند قرار گرفته است. اگر برچسب موضوع نتواند به درستی قسمتی از متن برنامه را آدرس دهی
کند، برچسب نادیده گرفته می‌شود و مستند به متن برنامه‌ای که بلافاصله بعد از آن آمده است 
نسبت داده می‌شود. در جدول \ref{جدول_برچسب_عنوان_QDoc} فهرستی از پرکاربرد ترین
برچسب‌های موضوع آورده شده است.
\begin{table}[ht]
 \centering
  {%
    \newcommand{\mc}[3]{\multicolumn{#1}{#2}{#3}}
    \begin{center}
    \begin{tabular}{|l|l|}\hline
      \mc{1}{r}{\lr{class}} & \mc{1}{r}{x}\\\hline
      \mc{1}{r}{\lr{enum}} & \mc{1}{r}{x}\\\hline
      \mc{1}{r}{\lr{example}} & \mc{1}{r}{x}\\\hline
      \mc{1}{r}{\lr{externalpage}} & \mc{1}{r}{x}\\\hline
      \mc{1}{r}{\lr{fn (function)}} & \mc{1}{r}{x}\\\hline
      \mc{1}{r}{\lr{group}} & \mc{1}{r}{x}\\\hline
      \mc{1}{r}{\lr{headerfile}} & \mc{1}{r}{x}\\\hline
      \mc{1}{r}{\lr{macro}} & \mc{1}{r}{x}\\\hline
      \mc{1}{r}{\lr{module}} & \mc{1}{r}{x}\\\hline
      \mc{1}{r}{\lr{namespace}} & \mc{1}{r}{x}\\\hline
      \mc{1}{r}{\lr{page}} & \mc{1}{r}{x}\\\hline
      \mc{1}{r}{\lr{property}} & \mc{1}{r}{x}\\\hline
      \mc{1}{r}{\lr{service}} & \mc{1}{r}{x}\\\hline
      \mc{1}{r}{\lr{typedef}} & \mc{1}{r}{x}\\\hline
      \mc{1}{r}{\lr{variable}} & \mc{1}{r}{x}\\\hline
    \end{tabular}
    \end{center}
  }%
 \caption{فهرست پر کاربرد ترین برچسب‌های عنوان در استاندارد \lr{QDoc}. }
 \label{جدول_برچسب_عنوان_QDoc}
\end{table}

\subsection{متن}
مستندهای متن، دسته‌ای از مستندها هستند که در ابزارهای خودکار مستند سازی قادر به تشخیص و ایجاد آنها نیستند.
برای نمون ابزارهای مستند ساز قادر به تشخیص مطلب‌های مرتبط با هم نیست. از این رو دسته‌ای از برچسب‌ها 
پیش بینی شده‌اند که در ایجاد و مدیریت اینگونه مستندها به کار می‌روند. به این دسته از برچسب‌ها برچسب‌های متن
گفته می‌شود. در جدول \ref{جدول_برچسب_متن_QDoc} فهرستی از پرکابرد ترین برچسب‌های متن آمده است.
\begin{table}[ht]
 \centering
  {%
    \newcommand{\mc}[3]{\multicolumn{#1}{#2}{#3}}
    \begin{center}
    \begin{tabular}{|l|l|}\hline
      \mc{1}{r}{\lr{compat}} & \mc{1}{r}{x}\\\hline
      \mc{1}{r}{\lr{contentspage}} & \mc{1}{r}{x}\\\hline
      \mc{1}{r}{\lr{indexpage}} & \mc{1}{r}{x}\\\hline
      \mc{1}{r}{\lr{ingroup}} & \mc{1}{r}{x}\\\hline
      \mc{1}{r}{\lr{inherits}} & \mc{1}{r}{x}\\\hline
      \mc{1}{r}{\lr{inmodule}} & \mc{1}{r}{x}\\\hline
      \mc{1}{r}{\lr{internal}} & \mc{1}{r}{x}\\\hline
      \mc{1}{r}{\lr{mainclass}} & \mc{1}{r}{x}\\\hline
      \mc{1}{r}{\lr{nextpage}} & \mc{1}{r}{x}\\\hline
      \mc{1}{r}{\lr{nonreentrant}} & \mc{1}{r}{x}\\\hline
      \mc{1}{r}{\lr{obsolete}} & \mc{1}{r}{x}\\\hline
      \mc{1}{r}{\lr{overload}} & \mc{1}{r}{x}\\\hline
      \mc{1}{r}{\lr{previouspage}} & \mc{1}{r}{x}\\\hline
      \mc{1}{r}{\lr{relates}} & \mc{1}{r}{x}\\\hline
      \mc{1}{r}{\lr{since}} & \mc{1}{r}{x}\\\hline
      \mc{1}{r}{\lr{startpage}} & \mc{1}{r}{x}\\\hline
      \mc{1}{r}{\lr{subtitle}} & \mc{1}{r}{x}\\\hline
      \mc{1}{r}{\lr{title}} & \mc{1}{r}{x}\\\hline
    \end{tabular}
    \end{center}
  }%
 \caption{فهرست پر کاربرد ترین برچسب‌های متن در استاندارد \lr{QDoc}. }
 \label{جدول_برچسب_متن_QDoc}
\end{table}


\subsection{نشانک‌ها}
آخرین دسته از برچسب‌ها، نشانک‌ها هستند. نشانک‌ها برای نوشتن و ساختاردهی کردن متن ایجاد شده
مورد استفاده قرار می‌گیرند. برای نمونه فهرست، جدول، پیونده به دیگر مستندها، ویا استفاده از 
یک تصویر خاص با استفاده از برچسب‌های نشانک ایجاد می‌شوند. در جدول \ref{جدول_برچسب_نشانک_QDoc}
فهرستی از پرکاربرد ترین نشانک‌ها در استاندارد \lr{QDoc} آمده است.
\begin{table}[ht]
 \centering
  {%
    \newcommand{\mc}[3]{\multicolumn{#1}{#2}{#3}}
    \begin{center}
    \begin{tabular}{|l|l|}\hline
      \mc{1}{r}{\lr{a}} & \mc{1}{r}{x}\\\hline
      \mc{1}{r}{\lr{bold}} & \mc{1}{r}{x}\\\hline
      \mc{1}{r}{\lr{brief}} & \mc{1}{r}{x}\\\hline
      \mc{1}{r}{\lr{c}} & \mc{1}{r}{x}\\\hline
      \mc{1}{r}{\lr{caption}} & \mc{1}{r}{x}\\\hline
      \mc{1}{r}{\lr{chapter}} & \mc{1}{r}{x}\\\hline
      \mc{1}{r}{\lr{code}} & \mc{1}{r}{x}\\\hline
      \mc{1}{r}{\lr{footnote}} & \mc{1}{r}{x}\\\hline
      \mc{1}{r}{\lr{image}} & \mc{1}{r}{x}\\\hline
      \mc{1}{r}{\lr{include}} & \mc{1}{r}{x}\\\hline
      \mc{1}{r}{\lr{input}} & \mc{1}{r}{x}\\\hline
      \mc{1}{r}{\lr{part}} & \mc{1}{r}{x}\\\hline
      \mc{1}{r}{\lr{quotation}} & \mc{1}{r}{x}\\\hline
      \mc{1}{r}{\lr{section1}} & \mc{1}{r}{x}\\\hline
      \mc{1}{r}{\lr{section2}} & \mc{1}{r}{x}\\\hline
      \mc{1}{r}{\lr{section3}} & \mc{1}{r}{x}\\\hline
      \mc{1}{r}{\lr{section4}} & \mc{1}{r}{x}\\\hline
      \mc{1}{r}{\lr{underline}} & \mc{1}{r}{x}\\\hline
    \end{tabular}
    \end{center}
  }%
 \caption{فهرست پر کاربرد ترین برچسب‌های نشانک در استاندارد \lr{QDoc}. }
 \label{جدول_برچسب_نشانک_QDoc}
\end{table}