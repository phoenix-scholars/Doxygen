% این قسمت که در واقع قسمتی از بخش استفاده از doxygen و doxywizard است به نحوه
% استفاده از doxygen از طریق خط فرمان می پردازد.
% محمد هادی منصوری ۹۰/۴/۲۵
\section{خط فرمان}

% استفاده از Doxygen به صورت خط فرمان
 همانگونه که در قسمت‌های قبل نیز اشاره شد، \lr{Doxygen} یک برنامه‌ی کاربردی بر
اساس خط فرمان است. اگر در خط فرمان دستور زیر را وارد کنید، که در واقع فراخوانی
\lr{Doxygen} با پارامتر \lr{--help} است، توضیح مختصری از نحوه‌ی استفاده از این
برنامه نمایش داده می شود.

\begin{Shell} 
doxygen --help
\end{Shell}

به طور کلی نحوه‌ی استفاده از این برنامه در خط فرمان به این صورت است: نوشتن
کلمه‌ی \lr{doxygen} و سپس نوشتن پارامترهای لازم برای کاری که می‌خواهیم انجام
دهیم. کلیه‌ی پارامترها به صورت یک یا چند کاراکتر که قبل از آن یک علامت منها (-)
قرار داده می شود، نوشته می‌شوند.

\lr{Doxygen} 
برای تولید مستند نیاز به یک پرونده‌ی پیکربندی دارد. در پرونده‌ی پیکربندی تنظیمات
مربوط به تولید مستندات تعیین می‌گردد. علاوه بر پرونده‌ی پیکربندی باید
پرونده‌هایی که حاوی کدها و مستندات هستند را نیز در اختیار این نرم‌افزار قرار داد
تا برای آن‌ها یک مستند یکپارچه تولید کند.

پس از اینکه مستندات خود را بر اساس قواعدی که در فصل‌های بعدی بیان شده است، 
نوشتید برای تولید مستند با استفاده از \lr{Doxygen} باید مراحل زیر را طی کنید:
\begin{enumerate}
\item ایجاد یک پرونده‌ی پیکربندی اولیه
\item ویرایش پرونده‌ی پیکربندی اولیه در صورت نیاز
\item اجرای نرم‌افزار \lr{Doxygen} و تولید مستند یکپارچه
\end{enumerate}

\subsection{ایجاد پرونده‌ی پیکربندی اولیه}

نوشتن پرونده‌ی پیکربندی به صورت دستی کاری زمانبر است اما برای اینکه ایجاد
پرونده‌ی پیکربندی ساده شود، نرم‌افزار \lr{Doxygen} می‌تواند یک پرونده‌ی پیکربندی
اولیه که حاوی مقادیر پیش‌فرض است، تولید کند. با دستور زیر می‌توان یک پرونده‌ی
پیکربندی اولیه را تولید کرد:

\begin{Shell}  
doxygen -g <config-name>
\end{Shell}

در دستور بالا \lr{<config-name>} نامی است که برای پرونده‌ی تولیدی تعیین می‌شود.
در صورتی که این پارامتر تعیین نشود، یعنی نامی برای پرونده‌ی تولیدی مشخص نشود،
پرونده‌ی تولیدی با نام \lr{Doxyfile} تولید می‌شود.
در صورتی که پرونده‌ای با نام \lr{configName} از قبل وجود داشته باشد،
\lr{Doxygen} ابتدا نام آن پرونده را به \lr{<configName>.bak} تغییر می‌دهد و پس
از آن پرونده‌ی پیکربندی را با نام داده شده تولید می‌کند.
اگر به جای پارامتر \lr{<config-name>} یک علامت منها (\-) قرار داده شود،
\lr{Doxygen} تلاش می‌کند که پرونده‌ی پیکربندی را از ورودی استاندارد \lr{(stdin)}
بخواند.

\subsection{ویرایش پرونده‌ی پیکربندی}

پرونده‌ی پیکربندی قالبی بسیار ساده دارد. این پرونده شامل تعدادی انتساب (تگ) به
یکی از دو صورت زیر است:

\begin{Config}
TAGNAME = VALUE
TAGNAME = VALUE1 VALUE2 ...
\end{Config}

حالت اول برای تنظیماتی است که فقط یک مقدار دارند و صورت دوم برای تنظیماتی است که
می‌توانند چند مقدار داشته باشند.

احتمالاً نیازی به تغییر مقدار اغلب این تگ‌ها ندارید و مقدار پیش‌فرض تعیین شده
برای آن‌ها در پرونده‌ی پیکربندی تولید شده را بدون تغییر خواهید گذاشت.
% برای اطلاعات بیشتر قسمت پیکربندی را مشاهده کنید.

در صورتی که نمی‌خواهید پرونده‌ی پیکربندی را با یک ویرایشگر متن ویرایش کنید،
می‌توانید از نرم‌افزار \lr{Doxywizard} استفاده کنید که یک واسط گرافیکی برای
تولید، خواندن و نوشتن پرونده‌های پیکربندی است و تعیین مقدار تنظیمات مختلف
پیکربندی را از طریق پنجره‌های گرافیکی فراهم می‌سازد. نحوه استفاده از
\lr{doxygen} از طریق واسط گرافیکی \lr{doxywizard} در قسمت‌های بعد شرح داده شده
است.

برای یک پروژه کوچک که شامل تعداد کمی پرونده‌ی منبع (\lr{source}) و سرآیند
(\lr{header}) به زبان \lr{C} و/یا \lr{C++} است، می‌توانید تگ \lr{INPUT} را بدون
مقدار رها کنید. در این حالت \lr{Doxygen} در مسیر جاری به دنبال پرونده‌های منبع
می‌گردد.

اگر پروژه‌ای دارید که شامل یک پوشه از پرونده‌های منبع است و یا شامل چندین پوشه
یا پوشه‌های تو در تو است، باید آدرس پوشه‌ها را در مقابل تگ \lr{INPUT} قرار دهید.
نوع پرونده‌ها یا به عبارتی قالب پرونده‌ها را باید در تگ \lr{FILE\_PATTERNS}
تعیین کنید (مثلا \lr{*.cpp *.h}). تنها پرونده‌هایی که با یکی از الگوهای داده شده
مطابقت داشته باشند توسط \lr{doxygen} تجزیه و تحلیل می‌شوند. در صورتی که در تگ
\lr{FILE\_PATTERNS} الگویی تعیین نشود، \lr{doxygen} فهرستی از پسوندهای مربوط به
پرونده‌های منبع را در نظر می‌گیرد. باید توجه داشت که \lr{doxygen} قالب پرونده‌ها
را از روی پسوند آنها تعیین می‌کند(مثلا \lr{.java} برای پرونده‌های منبع به زبان
جاوا، پسوند \lr{.cpp} برای زبان \lr{C++} و ...). در صورتی که پروژه شامل پوشه‌های
تو در تو است باید آدرس پوشه‌ی ریشه (اصلی) را به تگ \lr{INPUT} منتسب کنید و تگ
\lr{RECURSIVE} را نیز با \lr{YES} مقداردهی کنید. البته راه دیگر این است که آدرس
تمام پوشه‌ها (چه اصلی چه داخلی) را در تگ \lr{INPUT} به طور مستقیم اضافه کنید.

با استفاده از تگ‌های \lr{EXCLUDE} و \lr{EXCLUDE\_PATTERNS} می‌توانید پوشه‌ها یا
پرونده‌هایی را تعیین کنید که نباید تجزیه و تحلیل شوند. فرض کنید درون پوشه یا
پوشه‌هایی که آدرس آن‌ها را در تگ \lr{INPUT} قرار داده‌اید پوشه‌ها یا پرونده‌هایی
وجود دارند که نمی‌خواهید مستند آن‌ها به مستنداتی که توسط \lr{doxygen} تولید
می‌شود اضافه شود. برای این کار از دو تگ یاد شده استفاده می‌شود. در تگ
\lr{EXCLUDE} می‌توان پوشه‌هایی را تعیین کرد که ممکن است درون پوشه‌هایی که در
قسمت \lr{INPUT} تعریف شده‌اند قرار داشته باشند. به این ترتیب در هنگام تجزیه و
تحلیل پرونده‌های منبع، از این پوشه‌ها و پرونده‌های موجود در آن‌ها صرف نظر
می‌شود. با استفاده از تگ \lr{EXCLUDE\_PATTERNS} هم می‌توان الگوهایی را تعیین کرد
تا پرونده‌ها یا پوشه‌هایی که با یکی از این الگوها مطابقت دارند در هنگام تجزیه و
تحلیل در نظر گرفته نشوند. \lr{doxygen} این الگوها را با آدرس مطلق پرونده‌ها و
پوشه‌ها مقایسه می‌کند. به عنوان مثال برای جلوگیری از تجزیه و تحلیل تمام
پرونده‌هایی که در پوشه‌ای با نام \lr{inner} قرار دارند می‌توان به این صورت اقدام
کرد: \lr{EXCLUDE\_PATTERNS = */inner/*}.

\lr{doxygen} 
برای تشخیص اینکه یک پرونده را چگونه تجزیه و تحلیل کند به پسوند پرونده‌ها نگاه
می‌کند. اگر پرونده‌ای دارای پسوند \lr{.idl} یا \lr{.odl} باشد، آن را به عنوان یک
پرونده‌ی \lr{IDL} در نظر می‌گیرد. اگر پرونده‌ای پسوند \lr{.java} داشته باشد آن
را به عنوان یک پرونده که به زبان جاوا نوشته شده است در نظر می‌گیرد. پرونده‌هایی
که انتهای ‌آن‌ها \lr{.cs} است به عنوان پرونده‌های \lr{C\#} و پرونده‌هایی با
پسوند \lr{.py} به عنوان پرونده‌هایی به زبان پایتون \lr{(python)} در نظر گرفته
می‌شوند. در نهایت، پرونده‌هایی که پسوند آن‌ها \lr{.php}، \lr{.php4}، \lr{.inc}
یا \lr{.phtml} است به عنوان پرونده‌های منبع به زبان \lr{PHP} در نظر گرفته
می‌شوند. هر پسوند دیگر باعث می‌شود \lr{doxygen} آن پرونده را به عنوان پرونده‌ای
به زبان \lr{C/C++}، تجزیه و تحلیل کند (البته در صورتی که واقعا به زبان
\lr{C/C++} نوشته شده باشند). پرونده‌هایی که پسوند \lr{.m} دارند به عنوان
\lr{Objective-C} در نظر گرفته می‌شوند.

برای تعیین محلی که مستندات تولید شده توسط \lr{doxygen} باید در آنجا ذخیره شود از
تگ \lr{OUTPUT\_DIRECTORY} استفاده می‌شود. در مقابل این تگ باید نام یا آدرس
پوشه‌ای که مستندات تولیدی باید در آنجا قرار داده شوند نوشته شود.
آدرس داده شده می‌تواند نسبی یا مطلق باشد. آدرس نسبی نسبت به مسیر جاری که
\lr{doxygen} در آن اجرا می‌شود تعیین می‌گردد.
به عنوان مثال اگر در پرونده‌ی پیکربندی تگ \lr{OUTPUT\_DIRECTORY} را به صورت
\lr{OUTPUT\_DIRECTORY = result} مقداردهی کنید، آنگاه \lr{doxygen} در مسیر جاری
به دنبال پوشه‌ای به نام \lr{result} می‌گردد تا مستنداتی را که تولید می‌کند در آن
پوشه ذخیره کند.


اگر  پروژه‌ای دارید که هنوز مستندی که مناسب \lr{doxygen} باشد برای آن ننوشته‌اید
و یا مستندات هنوز کامل نیستند، در این حالت می‌توانید از \lr{doxygen} استفاده
کنید تا ساختار و شکل ظاهری مستندی که تولید می‌شود را مشاهده کنید. برای انجام این
کار باید تگ \lr{EXTRACT\_ALL} را در پرونده‌ی پیکربندی با \lr{YES} مقداردهی کنید.
به این ترتیب \lr{doxygen} مستندی شامل تمام موجودیت‌های پروژه (اعم از کلاس‌ها،
متدها، توابع و ...) تولید می‌کند، حتی اگر برای برخی موجودیت‌ها توضیحی نوشته نشده
باشد. به این نکته باید توجه داشت که موجودیت‌هایی که مستند نشده باشند (یعنی
توضیحی برای آن‌ها نوشته نشده باشد) در مستند تولید شده توسط \lr{doxygen} آورده
نمی‌شوند مگر اینکه تگ \lr{EXTRACT\_ALL} در پرونده‌ی پیکربندی با \lr{YES}
مقداردهی شده باشد. مقدار پیش‌فرض این تگ \lr{NO} است.

برای تحلیل قسمتی از یک نرم‌افزار که از قبل نوشته شده است می‌توان از \lr{doxygen}
کمک گرفت. برای تحلیل یک نرم‌افزار مفید است که برای هر موجودیت، ارجاعاتی به کد
مربوط به آن موجودیت در مستند وجود داشته باشد تا بتوان از طریق آن‌ها به کدهای
مربوطه دسترسی داشت. اگر تگ \lr{SOURCE\_BROWSER} با \lr{YES} مقداردهی شود،
\lr{doxygen} در مستنداتی که تولید می‌کند چنین ارجاعاتی را قرار می‌دهد. همچنین
می‌توان کاری کرد که متن کد مربوط به هر موجودیت به طور مستقیم در مستندات تولید
شده ظاهر شود. برای این کار باید تگ \lr{INLINE\_SOURCES} با \lr{YES} مقداردهی
شود. این کار می‌تواند برای مرور دستی کدهای نوشته شده مفید باشد. به عنوان مثال
فرض کنید می‌خواهید کدهایی را مرور کنید که توسط شخص دیگری نوشته شده است، یا اینکه
بخواهید کدها و مستنداتی که نوشته‌اید را در اختیار برنامه نویس دیگری قرار دهید تا
آن‌ها را مرور کند یا توسعه دهد. تولید اینگونه مستندات در چنین مواردی بسیار مفید
است.

\subsection{اجرای نرم‌افزار \lr{Doxygen} و تولید مستند}

پس از ایجاد و ویرایش پرونده پیکربندی، در این مرحله برای اجرای برنامه
\lr{doxygen} از طریق خط فرمان و تولید مستندات باید از دستور زیر استفاده شود:

\begin{Shell}
doxygen <config-name>
\end{Shell}

در دستور بالا \lr{<config-name>} نام پرونده‌ی پیکربندی است که از قبل تهیه شده است.


بسته به تنظیماتی که در پرونده‌ی پیکربندی قرار داده‌اید، \l{doxygen} یک یا چند
پوشه با نام‌های \lr{html}، \lr{rtf}، \lr{latex}، \lr{xml} و/یا \lr{man} را در
پوشه‌ای که به عنوان پوشه مقصد در تگ \lr{OUTPUT\_DIRECTORY} ذکر کرده‌اید
می‌سازد.این پوشه‌ها همانطور که از نامشان مشخص است حاوی مستندات تولید شده توسط
\lr{doxygen} در قالب‌های \lr{HTML}، \lr{RTF}، \lr{LATEX}، \lr{XML} و/یا
\lr{Unix-Man page} هستند.

مسیر خروجی پیش‌فرض برای ذخیره‌ی مستندات تولید شده، مسیری است که \lr{doxygen} در
آن اجرا شده است.
همانطور که در قسمت قبل گفته شد، مسیر یا پوشه‌ای که خروجی‌ها در آن قرار می‌گیرند
را می‌توان با مقداردهی به تگ \lr{OUTPUT\_DIRECTORY} تغییر داد. نام پوشه‌ی خاص هر
قالب خروجی را هم می‌توان با مقداردهی به تگ‌های \lr{HTML\_OUTPUT}،
\lr{RTF\_OUTPUT}، \lr{LATEX\_OUTPUT}، \lr{XML\_OUTPUT} و/یا \lr{MAN\_OUTPUT} در
پرونده‌ی پیکربندی تغییر داد. در صورتی که پوشه‌ی مربوط به مسیر خروجی تعیین شده
وجود نداشته باشد، \lr{doxygen} سعی می‌کند آن پوشه را بسازد و بعد مستندات تولیدی
را در آن قرار دهد. البته باید توجه داشت که \lr{doxygen} کل پوشه‌های مذکور در
مسیر را نمی‌سازد و فقط داخلی‌ترین پوشه را در صورت عدم وجود می‌سازد. سایر
پوشه‌های پدر در مسیر خروجی باید از قبل وجود داشته باشند.

\subparagraph{خروجی \lr{HTML}}

مستندات تولید شده در قالب \lr{HTML} را می‌توان به وسیله‌ی یک مرورگر \lr{HTML}
باز کرده و مشاهده کرد. صفحه‌ی اصلی (یا صفحه‌ی آغازین) مستندات در این قالب
پرونده‌ی \lr{index.html} در پوشه‌ی \lr{html} است. برای مشاهده‌ی بهترین نتیجه از
این مستندات باید از مرورگری که از \lr{CSS}ها پشتیبانی می کند برای باز کردن و
مشاهده این مستندات استفاده شود (نمایش مستندات تولید شده در قالب \lr{HTML} با
استفاده از مرورگرهای \lr{Mozilla}، \lr{Safari}، \lr{Konqueror}، و \lr{IE6} مناسب
است).

در مستندات \lr{HTML} می‌توان یک فهرست درختی یا نمایی درختی از مطالب موجود در
مستند وارد کرد. برای این کار باید تگ \lr{GENERATE\_TREEVIEW} در پرونده‌ی
پیکربندی با \lr{YES} مقداردهی شود. البته در صورت تولید مستندات با چنین امکانی،
برای مشاهده‌ی آن‌ها نیاز به مرورگرهایی دارید که علاوه بر \lr{CSS}، از \lr{DHTML}
و \lr{JavaScript} نیز حمایت کنند.
تقریباً تمام مرورگرهای جدید از این امکانات حمایت می‌کنند و مشکلی با نمایش
اینگونه مستندات ندارند.


\subparagraph{خروجی \lr{LaTeX}}

مستند \lr{LaTeX} تولید شده توسط \lr{doxygen} ابتدا باید به وسیله‌ی یک مترجم
(کامپایلر) \lr{LaTeX} ترجمه شود. \lr{doxygen} برای ساده سازی روند ترجمه‌ی
مستندات تولید شده یک پرونده‌ی \lr{Makefile} را در پوشه‌ی \lr{LaTeX} قرار می‌دهد.

محتویات پرونده‌ی \lr{Makefile} به مقدار تگ \lr{USE\_PDFLATEX} در پرونده‌ی
پیکربندی بستگی دارد. اگر این تگ غیر فعال شده باشد (یعنی با \lr{NO} مقداردهی شده
باشد)، با اجرای دستور \lr{make} در مسیری که مستندات در
 قالب \lr{LaTeX} تولید شده‌اند، یک پرونده‌ی \lr{dvi} با نام \lr{refman.dvi}
 تولید خواهد شد. این پرونده را می‌توان با
استفاده از \lr{xdvi} مشاهده کرد و یا با اجرای دستور \lr{make ps} به یک پرونده‌ی
\lr{PostScript} تبدیل کرد (این کار نیاز به \lr{dvips} دارد). این پرونده را
می‌توان به \lr{PDF} هم تبدیل کرد، البته در صورتی که مفسر \lr{ghostscript} نصب
شده باشد. برای تبدیل به \lr{PDF} کافی است دستور \lr{make pdf} را اجرا کنید.

برای اینکه خروجی \lr{PDF} بهتری داشته باشید باید تگ‌های \lr{PDF\_HYPERLINK} و
\lr{USE\_PDFLATEX} در پرونده پیکربندی را با \lr{YES} مقداردهی کنید. در این صورت
پرونده‌ی \lr{Makefile} موجود در پوشه‌ی مستندات \lr{LaTex}، تنها حاوی دستوراتی
برای ایجاد \lr{refman.pdf} خواهد بود.

\subparagraph{خروجی \lr{RTF}}

\lr{Doxygen} 
برای تولید مستندات در قالب \lr{RTF} تنها یک پرونده به نام \lr{refman.rtf} تولید
می‌کند و تمام مستندات را در همان یک پرونده قرار می‌دهد. این پرونده برای وارد
کردن (\lr{import}) در مایکروسافت وورد (\lr{Microsoft Word}) بهینه شده است.
برخی اطلاعات با استفاده از \lr{field} کدگذاری شده‌اند. برای نمایش مقدار واقعی
باید تمام آن‌ها را انتخاب کنید (یعنی گزینه‌ی \lr{Edit-> Select all} را بزنید) و
سپس \lr{fielf} را عوض کنید (با کلیک راست و انتخاب گزینه‌ی \lr{option} از منویی
که ظاهر می‌شود).

\subparagraph{ خروجی \lr{XML} }

خروجی \lr{XML} تولیدی توسط \lr{doxygen} شامل مجموعه‌ای ساختار یافته از اطلاعاتی
است که به وسیله‌ی \lr{doxygen} جمع‌آوری می‌شود.
هر مولفه (کلاس، فضای نام، پرونده و ...) پرونده‌ی \lr{xml} مربوط به خود را دارد.
علاوه بر آن‌ها یک پرونده‌ی شاخص به نام \lr{index.xml} نیز وجود دارد.
 
یک پرونده به نام \lr{combine.xslt} نیز تولید می‌شود که می‌تواند برای ترکیب تمام
پرونده‌های \lr{xml} در یک پرونده به کار رود.

\subparagraph{خروجی \lr{Man page}}

در مورد مستندات تولیدی در قالب \lr{Man-page} باید به این نکته توجه داشت که این
قالب محدودیت‌های محدودی دارد. بنابراین برخی اطلاعات (مثل دیاگرام کلاس‌ها،
ارجاعات و فرمول‌ها) از دست خواهند رفت.
