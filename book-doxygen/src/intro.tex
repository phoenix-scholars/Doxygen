%
% حق نشر 1390-1402 دانش پژوهان ققنوس
% حقوق این اثر محفوظ است.
% 
% استفاده مجدد از متن و یا نتایج این اثر در هر شکل غیر قانونی است مگر اینکه متن حق
% نشر بالا در ابتدای تمامی مستندهای و یا برنامه‌های به دست آمده از این اثر
% بازنویسی شود. این کار باید برای تمامی مستندها، متنهای تبلیغاتی برنامه‌های
% کاربردی و سایر مواردی که از این اثر به دست می‌آید مندرج شده و در قسمت تقدیر از
% صاحب این اثر نام برده شود.
% 
% نام گروه دانش پژوهان ققنوس ممکن است در محصولات دست آمده شده از این اثر درج
% نشود که در این حالت با مطالبی که در بالا اورده شده در تضاد نیست. برای اطلاع
% بیشتر در مورد حق نشر آدرس زیر مراجعه کنید:
% 
% http://dpq.co.ir/licence
%
% در مقدمه کتاب ابتدا در مورد پروژه های بزرگ و نیاز آنها به مستندسازی کدهای
% نوشته شده و همچنین نیاز ضروری کد مربوط به کتابخانه های کاربردی به مستندات صحبت
% می شود. پس از آن روش ها تولید مستند و مشکلات مستندسازی مطرح می شود و بعد چند
% ابزار برای مستندسازی خودکار معرفی شده و سپس داکسی ژن به عنوان ابزاری برای
% مستندسازی خودکار معرفی می شود و همچنین داکسی ویزار نیز به عنوان واسط گرافیکی
% داکسی ژن معرفی می شود. علاوه بر این سعی شود داکسی ژن با سایر ابزارهای خودکار
% تولید مستند مقایسه شود.
% پس از آن به معرفی بخش های مختلف کتاب پرداخته شده و خلاصه ای در مورد اینکه در
% هر بخش و فصل چه چیزی آورده شده. در پایان نیز مخاطبین کتاب و نحوه مطالعه کتاب
% آورده می شود.
% %%%%%%%%%%%%%%%%%%%%%% ایجاد یک مستند مناسب از واسطه‌های
% برنامه‌سازی\footnote{\lr{API}} و گردآوری مستندهای متفاوت ایجاد شده از بسته‌های
% نرم‌افزاری حجیم، چالش‌های متفاوتی را برای توسعه دهندگان نرم‌افزار ایجاد کرده
% است.
% می‌توان در ادامه چندتا از این مشکلها را نام برد

%برای غلبه بر این دسته از مشکلات، ابزارهای خودکار متفاوتی توسعه یافته اند. از این دسته ابزارهای می‌توان به 
%مواردی چون 
%\lr{JavaDoc}\cite{javadocsun}
%،\lr{QtDoc}
%و
%\lr{Doxygen}
%اشاره کرد.
%%%%%%%%%%%%%%%%%%%%%%%

%
% حق نشر 1390-1402 دانش پژوهان ققنوس
% حقوق این اثر محفوظ است.
% 
% استفاده مجدد از متن و یا نتایج این اثر در هر شکل غیر قانونی است مگر اینکه متن حق
% نشر بالا در ابتدای تمامی مستندهای و یا برنامه‌های به دست آمده از این اثر
% بازنویسی شود. این کار باید برای تمامی مستندها، متنهای تبلیغاتی برنامه‌های
% کاربردی و سایر مواردی که از این اثر به دست می‌آید مندرج شده و در قسمت تقدیر از
% صاحب این اثر نام برده شود.
% 
% نام گروه دانش پژوهان ققنوس ممکن است در محصولات دست آمده شده از این اثر درج
% نشود که در این حالت با مطالبی که در بالا اورده شده در تضاد نیست. برای اطلاع
% بیشتر در مورد حق نشر آدرس زیر مراجعه کنید:
% 
% http://dpq.co.ir/licence
%
\chapter{مقدمه}

یکی از مهمترین چالش‌ها در تولید و توسعه نرم‌افزارهای بزرگ، مستندسازی آن‌هاست.
مستندسازی در پروژه‌های نرم‌افزاری بزرگ بسیار ضروری است. به خصوص پروژه‌هایی که
پیاپی در حال توسعه و تولید نسخه‌های جدیدتر هستند و همچنین پروژه‌هایی که زمانی
طولانی صرف ساخت آن‌ها می‌شود و در این مدت ممکن است تیم‌ها و افراد مختلفی ادامه
پروژه را بر عهده گیرند.

فرض کنید پروژه‌ای شامل چندین فاز باشد و فازهای اولیه آن توسط افراد و گروه‌هایی
انجام شده باشد. حال تصمیم گرفته شده تا افراد یا گروه‌های جدید به پروژه اضافه
شوند. یا اینکه فرض کنید چند نسخه از یک نرم‌افزار تولید شده و برای تولید نسخه‌های
جدید، گروه‌های دیگری مسئولیت این کار را بر عهده گرفته‌اند. اگر مستند مناسب و
کاملی از پروژه و به خصوص از نحوه پیاده‌سازی موجود نباشد توسعه پروژه یا تولید
نسخه‌های جدیدتر بسیار طاقت فرسا، زمان‌بر و گاهی غیر ممکن است و این موضوع علاوه بر هدر
رفتن زمان باعث تحمیل هزینه‌های اضافی نیز می‌شود.

بیشتر برنامه‌نویسان علاقه‌ای به کند و کاو در کدهایی که توسط دیگران نوشته شده است
ندارند. اغلب، برای یک برنامه‌نویس کابوسی وحشتناک است که بخواهد کدهایی که توسط
دیگران نوشته شده است را بخواند و از طریق آن‌ها مدل و منطق برنامه را بفهمد! به
خصوص اگر حجم کد برنامه مذکور زیاد باشد (مثلا ۷ یا ۸ هزار خط یا بیشتر!). حتی اگر
مدل مربوط به یک نرم‌افزار در دسترس باشد باز هم وجود مستندات مربوط به کدها کمک
شایانی به توسعه نرم‌افزار خواهد کرد.
علاوه بر این‌ها، در برنامه‌های کاربردی مثل کتابخانه‌های کاربردی، در صورتی که
مستندی برای معرفی کلاس‌ها و متدهای آن و نحوه استفاده از آن‌ها در دسترس نباشد، به
کار بردن آن‌ها امکان‌پذیر نخواهد بود. در واقع می‌توان گفت کتابخانه کاربردی نوشته
شده بدون مستند، تفاوتی با نوشته نشده آن ندارد!

حال فرض کنیم قرار باشد برای یک نرم‌افزار مستند فنی نوشته شود. یا برای کدهای آن
توضیحات و مستندات تهیه شود.
خب سوال اول این است که چه کسی باید این‌گونه مستندات را بنویسد؟ سوال دیگر اینکه
کجا و در چه قالبی بنویسد؟ شکی نیست که بهترین فرد برای تهیه مستندات فنی و مستندات
مربوط به کدها شخص برنامه‌نویس است. اینکه یک برنامه‌نویس برای کدهایی که خود نوشته
مستندات لازم را تهیه کند مطمئنا بهتر از این است که شخص دیگری بخواهد برای آن کدها
مستند فنی تهیه کند. علاوه بر آن تهیه مستند و توضیح برای کدها توسط شخص دیگری غیر
از برنامه‌نویس، باز هم همان حکایت کند و کاو در کدهاست. در پاسخ به سوال دوم نیز
باید گفت بهترین محل برای نوشتن مستندات و توضیحات مربوط به کدها، لابه‌لای همان کد
است. تقریبا در تمام زبان‌های برنامه‌نویسی روشی برای نوشتن توضیح وجود دارد.
استفاده از این امکان بهترین راه برای مستندنویسی فنی است چرا که به برنامه‌نویس
اجازه می‌دهد در همان لحظه که کد را می‌نویسد توضیحات و مستند مربوط به آن را نیز
در کنارش بنویسد.


با توجه اهمیت مستندسازی، ابزارهای مختلفی برای تهیه مستند از کدهای نوشته شده به
طور خودکار، به وجود آمده‌اند. این‌گونه ابزارها اغلب به برنامه‌نویسان این امکان
را می‌دهند تا هنگام برنامه‌نویسی مستندات لازم را نیز همراه کدها بنویسند و نگران
تولید مستند نباشند. می‌توان با استفاده از این ابزارها مستندات نوشته شده در کدها
را استخراج کرده و به صورت یک مستند یکپارچه تهیه کرد. این مستندات را می‌توان به
گروه‌های بعدی برای توسعه نرم‌افزار ارائه داد. در مورد کتابخانه‌های کاربردی نیز
می‌توان این مستندات را به عنوان راهنمای استفاده در دسترس استفاده‌کنندگان از
کتابخانه قرار داد.

\begin{sloppypar}
از جمله ابزارهای تولید خودکار مستند می‌توان به 
\lr{JavaDoc}، \lr{QtDoc}، \lr{classdoc}، \lr{DOC++}، \lr{ROBODoc} و \lr{Doxygen} 
اشاره کرد. این ابزارها می‌توانند پرونده‌های حاوی کدها را دریافت کرده و مستندات نوشته شده در آن‌ها 
را استخراج کرده و یک مستند یکپارچه تولید کنند. کلیه ابزارهای نام برده رایگان و متن باز بوده و 
تحت پیمان‌نامه عمومی موسسه \lr{GNU} یعنی \lr{GPL} هستند. 
همه آن‌ها تحت سیستم عامل های مختلف ویندوز، لینوکس، یونیکس، \lr{Mac} و \lr{BSD} قابل اجرا هستند. 
\end{sloppypar}

ابزارهای \lr{Javadoc} و \lr{classdoc} مخصوص زبان جاوا هستند. \lr{QtDoc} ویژه
زبان \lr{C/C++} است و ابزار \lr{DOC++} برای زبان‌های جاوا، \lr{C/C++} و \lr{IDL}
قابل استفاده است. دو ابزار \lr{Doxygen} و \lr{ROBODoc} نسبت به بقیه، زبان‌های
برنامه‌نویسی بیشتری را پوشش می‌دهند. این دو ابزار تقریبا تمام زبان‌های
برنامه‌نویسی متداول امروزی را حمایت می‌کنند. همچنین می‌توانند مستندات را در
قالب‌های مختلفی تولید کنند (که مهمترین آن‌ها \lr{HTML} و \lr{PDF} است).
اما \lr{Doxygen} نسبت به \lr{ROBODoc} یک برتری مهم دارد. \lr{Doxygen} از یونیکد
حمایت می‌کند در حالی که \lr{ROBODoc} تنها از اسکی پشتیبانی می‌کند. بنابراین 
می‌توانید مستندات را به هر زبانی (از جمله فارسی) در پرونده‌های منبع نوشته و با
استفاده از \lr{Doxygen} مستند مورد نظر خود را تولید کنید.

\begin{sloppypar}
در این کتاب به معرفی \lr{Doxygen} و نحوه نصب و استفاده از آن پرداخته شده است. زبان‌هایی که تا زمان چاپ این کتاب 
توسط \lr{Doxygen} حمایت می‌شوند عبارتند از: جاوا، 
\lr{C}، \lr{C++}، \lr{C\#}، \lr{Objective-C}، \lr{PHP}، \lr{Python}، \lr{Fortran}، \lr{IDL}، \lr{VHDL} و \lr{D}. 
علاوه بر این با استفاده از این ابزار می‌توان مستندات را در قالب‌های 
مختلفی چون \lr{HTML}، \lr{PDF} یا حتی \lr{\LaTeX} تولید کرد. \lr{Doxygen} از طریق خط فرمان قابل استفاده است و 
ظاهر گرافیکی ندارد. اما ابزار دیگری تحت عنوان \lr{Doxywizard} وجود دارد که یک واسط گرافیکی 
برای استفاده از \lr{Doxygen} است. این برنامه نیز رایگان است و تحت پیمان‌نامه \lr{GPL} قرار دارد. در کتاب حاضر 
علاوه بر \lr{Doxygen}، استفاده از \lr{Doxywizard} نیز شرح داده شده است.
\end{sloppypar}

%
% حق نشر 1390-1402 دانش پژوهان ققنوس
% حقوق این اثر محفوظ است.
% 
% استفاده مجدد از متن و یا نتایج این اثر در هر شکل غیر قانونی است مگر اینکه متن حق
% نشر بالا در ابتدای تمامی مستندهای و یا برنامه‌های به دست آمده از این اثر
% بازنویسی شود. این کار باید برای تمامی مستندها، متنهای تبلیغاتی برنامه‌های
% کاربردی و سایر مواردی که از این اثر به دست می‌آید مندرج شده و در قسمت تقدیر از
% صاحب این اثر نام برده شود.
% 
% نام گروه دانش پژوهان ققنوس ممکن است در محصولات دست آمده شده از این اثر درج
% نشود که در این حالت با مطالبی که در بالا اورده شده در تضاد نیست. برای اطلاع
% بیشتر در مورد حق نشر آدرس زیر مراجعه کنید:
% 
% http://dpq.co.ir/licence
%
\section{سازمان مطالب کتاب} 

کتاب در سه بخش مجزا تدوین شده که در هر یک هدف خاصی را دنبال می‌شود.
بخش اول کتاب نحوه نصب این ابزارها و چگونگی استفاده از قابلیت‌های مختلف آن‌ها
توضیح داده شده است.
این بخش شامل دو فصل \emph{نصب} و \emph{استفاده} می‌باشد. در فصل نصب، نحوه نصب
این ابزارها روی  سیستم‌عامل‌های لینوکس و ویندوز شرح داده شده است.

در بخش دوم نحوه مستندنویسی بیان شده است. اینکه چگونه مستندات خود را لابه‌لای
کدها بنویسیم تا \lr{Doxygen} بتواند آن‌ها را تشخیص دهد. در فصل‌های مختلف این بخش
قابلیت‌های مختلفی که می‌توان در مستندنویسی به کار برد بیان شده است، از جمله نحوه
ایجاد فهرست، نحوه فرمول نویسی و غیره.

در بخش سوم بر اساس تجاربی که در پروژه‌های برنامه‌نویسی و مستند کردن آن‌ها به دست
آورده‌ایم الگویی برای مستندنویسی ارائه داده‌ایم. اینکه مستندات را چگونه و کجا
بنویسیم تا مستند تولیدی توسط \lr{Doxygen} کامل و مناسب باشد و خوانندگان بتوانند
به راحتی از مستند استفاده کنند. و اینکه چگونه مستندنویسی کنیم و چه بنویسیم تا
مستندات نه بیش از اندازه زیاد و کلافه کننده باشند و نه چنان خلاصه که کاربر
نتواند اطلاعات مورد نیاز خود را در آن بیابد.

اساسی‌ترین هدفی که همواره تیم‌های توسعه به دنبال آن هستند، خودکارسازی فرآیند
توسعه و نگداری سیستم‌ها است. در این زمینه کارهای متنوعی انجام شده و در بسیاری
موارد ابزارهای قدرتمندی در این زمینه ارائه شده است. دور از انتظار نیست که بتوان
فرآیند تولید مستند فنی یک سیستم را نیز به صورت خودکار انجام داد. در بخش‌های
پایانی این کتاب ابزارها و روش‌های متنوعی برای خودکار سازی فرآیند تولید مستند
ارائه شده است که می‌تواند در افزایش بهروری تیم‌های توسعه مفید باشد.

% TODO: maso 1391: تکمیل نمونه‌ها
% در بخش چهارم نیز نمونه‌هایی از نحوه مستندنویسی در کدها و مستندات  و نمونه‌هایی
% از مستندات تولید شده به وسیله \lr{Doxygen} آورده شده است. در حقیقت این بخش شامل
% دو پروژه نمونه است. یکی به زبان جاوا و دیگری به زبان \lr{C}.

در این کتاب تلاش شده است که بخش‌ها و فصل‌های ایجاد شده به گونه‌ای باشد که کمترین
وابستگی را به یکدیگر داشته باشند. با این وجود نمی‌توان مدعی بود که هیچ وابستگی
میان فصل‌ها وجود ندارد. در شکل \ref{image/struct}
وابستگی میان فصل‌ها نمایش داده شده است.

\begin{figure}
\centering
\includegraphics[width=.8\textwidth]{image/struct}
\label{image/struct}
\end{figure}

% TODO: maso 1391: توصیف پیوستگی میان فصلها
% در اینجا باید وابستگی میان فصل‌ها به صورت کامل تشریح شود.

\section{مخاطبین}

این کتاب مناسب کلیه برنامه‌نویسان، طراحان و ناظران پروژه‌های نرم‌افزاری است.
همچنین برای دانشجویان و مهارت‌آموزانی که علاقه دارند در مورد مستندنویسی و نحوه
تولید مستند و راهنما در پروژه‌های نرم‌افزاری و کتابخانه‌های کاربردی اطلاعاتی کسب
کنند و احتمالا آن‌ها را به کار گیرند نیز مناسب است. و به طور خاص برای افرادی که
وظیفه‌شان مستندسازی است، کتاب حاضر می‌تواند مفید باشد.

از آنجا که در این کتاب نه تنها شامل چگونگی نوشتن مستند بلکه تجربیات ما در زمینه
مستند سازی تکنیکی است، می‌تواند به عنوان یک نقطعه شروع برای تیم‌های توسعه
نرم‌افزاری در نظر گرفته شود. خوانندگان نه تنها با اصول نوشتن مستند بلکه با
ابزارهای مناسب برای تولید و خودکارسازی فرآیند مستند سازی اشنا خواهند شد.


