%
% حق نشر 1390-1402 دانش پژوهان ققنوس
% حقوق این اثر محفوظ است.
% 
% استفاده مجدد از متن و یا نتایج این اثر در هر شکل غیر قانونی است مگر اینکه متن حق
% نشر بالا در ابتدای تمامی مستندهای و یا برنامه‌های به دست آمده از این اثر
% بازنویسی شود. این کار باید برای تمامی مستندها، متنهای تبلیغاتی برنامه‌های
% کاربردی و سایر مواردی که از این اثر به دست می‌آید مندرج شده و در قسمت تقدیر از
% صاحب این اثر نام برده شود.
% 
% نام گروه دانش پژوهان ققنوس ممکن است در محصولات دست آمده شده از این اثر درج
% نشود که در این حالت با مطالبی که در بالا اورده شده در تضاد نیست. برای اطلاع
% بیشتر در مورد حق نشر آدرس زیر مراجعه کنید:
% 
% http://dpq.co.ir/licence
%
\section{مستند پیاده سازیی}
همان گونه که پیش از نیز بیان شد، علاوه بر مستند فنی دسته‌ای دیگر از مستندها وجود دارد که در مورد 
پیاده سازی سیستم است. در این مستند برخلاف مستند فنی به روش مورد استفاده در پیاده سازی متدها و 
الگوریتم‌ها ذکر می‌شود تنها در توسعه سیستم مورد استفاده است. گرچه مستند پیاده سازی از اهمیت ویژه‌ای
برخوردار است، با این وحود نباید در مستند فنی ظاهر شود.

فرض کنید که یک توسعه دهنده سیستم در زمان پیاده سازی به یک ایده جدید در مورد پیاده سازی سیستم 
رسید است. این ایده یک مستند بسیار مهم است که باید در سیستم نگه داشته شود (تا جایی که بسیاری
از متدلوژی‌ها بر اساس این ایده‌ها سازمان‌دهی می‌شوند). مستندهای از این دست، کاملا در مورد پیاده‌سازی
سیستم است از این رو کاربردی برای کاربران سیستم ندارند.

مستندهای پیاده سازی در لابلای کدها نوشته می‌شود و از ساختار سازمان یافته‌ای همانند مستند فنی برخوردار
نیستند. گرچه چگونگی نوشتن مستند پیاده سازی کاملا شخصی است با این وجود دسته‌ای از استاندارهای برای سازمان
دهی آن وچود دارد که در ادامه به آن پرداخته خواهد شد.
مستند پیاده سازی به دو روش در برنامه‌ها بیان می‌شود: چند و تک خطی. زمانی که مستند پیاده سازی طولانی
است به صورت چند خط نوشته می‌شود این مستند به صورد زیر نوشته می‌شود:
 
\begin{latin}
\lstset{language=C++}
\begin{lstlisting}[frame=single] 
/*
 * Implementation document
 * this document is structed in multi line.
 */
\end{lstlisting}
\end{latin}

البته مستندهای کوتاه تنها در یک خط و به صورت زیر در متن برنامه‌ها نوشته می‌شود:

\begin{latin}
\lstset{language=C++}
\begin{lstlisting}[frame=single] 
// Implemtnation Doceumtn
\end{lstlisting}
\end{latin}

هر قسمت مستند که به صورت یکی از دو روش بالا در متن برنامه‌ها نوشته شود به عنوان یک مستند پیاده سازی
در نظر گرفت شده و در مستند فنی تولید شده وارد نخواهد شد.

\subsection{برچسب‌های مستند پیاده‌سازی}
فرض کنید که یک گروه پیاده‌سازی می‌خواهد با استفاده از یک ماشین برنامه ایجاد شده را به صورد کامل
بررس کرده و بر اساس آن پیام‌های مناسبی را در جهت بهبود و گسترش سیستم، ایجاد کند. بی شک این ماشین
باید مستندهای پیاده سازی را بررسی کرده و پیام‌های خود را مبتنی بر آنها ایجاد کند. برای نمونه انتظار
می‌رود که برنامه تعیین کند که کدام قسمت‌ها از برنامه ایجاد شده نیاز به بازنویسی، بهبود و یا حذف شدن 
دارد تا بتواند در فرآیند توسعه مورد استفاده قرار گیرد. بی شک این ماشین بدون در نظر گرفتن استاندارهای
مناسب برای نوشتن مستند پیاده سازی نمی‌تواند پیام‌های مناسبی را ایجاد کند. در این قسمت یک ساختار ابتدایی
برای نوشتن مستند پیاده سازی تشریح خواهد شد که می‌تواند این فرآیند را به صورت کامل پوشش دهد.

برخلاف مستند فنی، در مستند پیاده‌سازی تنها تعداد محدودی برچسب موجود است که برای ساختاردهی مستند مورد
استفاده قرار می‌گیرد. این برچسب‌ها باید هموار در ابتدای نخستن خط از هر بسته مستند پیاده سازی نوشته
شده و به صورت کامل از ساختار تعریف شده پیروی کند. گرچه امکان نوشتن مستند پیاده‌سازی در چندین خط وجود
دارد اما همواره در سطر اول باید با استفاده از یک جمله به صورت کامل و شفاف هدف اصلی مستند نوشته شود.
علاوه بر این باید تعیین شود که چه فردی و در چه زمان مستند را ایجاد و یا ویرایش کرده است. مستند زیر
را در نظر بگیرد:


\begin{latin}
\lstset{language=C++}
\begin{lstlisting}[frame=single] 
/*
 * TODO : maso 1390-12 : Document title.
 * document text.
 */
\end{lstlisting}
\end{latin}

همان گونه که در این قطعه مستند قابل مشاهد است، در سطر نخست با استفاده از یک برچسب، نام توسعه دهنده، 
تاریخ و عنوان نه تنها مفاهیم کلی مستند انتقال یافته بلکه امکان ردیابی اطلاعات نیز ممکن شده است. 
برای سادگی و ایجاد تمایز میان برچسب‌های مورد استفاده در مستند فنی و پیاده سازی، برچسب‌های مورد استفاده
در مستند پیاده‌سازی را چمپ\LTRfootnote{برچسب مستند پیاده‌سازی} 
 گفته خواهد شد. ساختار کلی چمپ‌ها به صورت زیر است:


\begin{latin}
\lstset{language=C++}
\begin{lstlisting}[frame=single] 
/*
 * <TAG> : <Developer> <Date> : <Short Discription>
 * <Discription>
 */
\end{lstlisting}
\end{latin}

گاهی کل مستند فنی در یک سطر قابل بیاد است در این صورت می‌توان از مستند پیاده‌سازی تک سطری استفاده کرد.
قالب کلی این مستند نیز با استفاده از چمپ‌ها به صورت زیر است:

\begin{latin}
\lstset{language=C++}
\begin{lstlisting}[frame=single] 
//<TAG> : <Developer> <Date> : <Short Discription>
\end{lstlisting}
\end{latin}


\subsubsection{\lr{FIXME}}
این چمپ از بالاترین اولویت برخورد دارد است. زمانی که یک برنامه ناقص است و بر این اساس سیستم قابلیت
رسیدن به اهداف پیش بینی شده را ندارد از این برچسب استفاده می‌شود. به بیان دیگر هر گاه قسمتی از برنامه
نوشته نشده است و باید پیش از اتمام پروژه حتما کامل شود از این برچسب استفاده می‌شود.

\subsubsection{\lr{TODO}}
اولویت این چمپ معمولی است. زمانی که انجام یک کار در پیاده‌سازی کامل یک سیستم لازم است اما عدم انجام 
آن آسیب مهمی را به سیستم وارد نمی‌کند از این چمپ استفاده می‌شود. برای نمونه فرض کنید که ممکن است در
برخی از مواقع شبکه مورد استفاده دچار مشکل شود در این صورت باید برنامه‌ای نوشته شود که بتواند به صورت
مناسب مشکل را به کاربر انتقال دهد اما احتمال این رویداد در سیستم بسیار کم است. بنابر این می‌توان 
با استفاده از این چمپ پیاده‌سازی برنامه مناسب را به زمانی دیگر موکول کرد.

\subsubsection{\lr{WARNING}}
این چمپ از پایین‌ترین اولویت برخوردار است و تنها زمانی به کار می‌رود که نیاز به گوش زد کردن برخی از 
نکات وجود داشته باشد. برای نمونه فرض کنید که نیاز است داده‌های ورودی یک پردازش داده‌های مثبت باشند
اما بررسی این نکته که آیا تمام داده‌ها مثبت است در برنامه انجام نشده باشد (که منشا آن می‌تواند ساختار
داده‌ای موجود در سیستم باشد). اما این کار از امنیت سیستم می‌کاهد چرا که ورود یک داده نا معتبر می‌تواند
کار سیستم را متوقف کند. از این رو با استفاده از این چمپ می‌توان مشکل را با توسعه دهندگان بعد گوش زد
کرد.

\subsubsection{\lr{MINDSTORME}}
گرچه اولویت این چمپ با \lr{WARNING} برابر است اما مفهومی کاملا متفاوت دارد. فرض کنید که در طی فرآیند 
توسعه سیستم ایده‌هایی جدید برای بهبود و گسترش سیستم به وجود آید. در این صورت باید راهکاری برای 
جمع آوری این ایدها و انتقال آنها به گروهای بعدی توسعه ایجاد شود. با استفاده از این چمپ می‌توان ایده‌های
جدید برای توسعه نرم‌افزار را در خلال آن ایجاد و مدیریت کرد.

