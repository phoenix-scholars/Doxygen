

\section{مدیریت بسته کلاه قرمز}


% در این قسمت باید در مورد این سیستم مدیریت بسته به صورت کامل بحث شود
\lr{RPM} یا مدیریت بسته کلاه قرمز\LTRfootnote{\Gls{red hat package manager}} یک
مجموعه ابزار نرم‌افزاری برای ایجاد و نصب و راه اندازی سیستم‌های نرم‌افزاری است. می‌توان
گفت که پیشینه \lr{RPM} به صورت جدایی ناپذیری با پیشینه سیستم عامل لینوکس در هم
آمیخته است، از این رو بررسی پیشینه‌ لینوکس مفید است. لینوکس یک پیاده سازی کامل
از سیستم‌های عمل شبیه \lr{Unix} است که مانند یک طوفان دنیای محاسبات رایانه‌ای را
درنبردید.

هم زمان با توسعه و گسترش یافتن لینوکس، رایانه‌های شخصی تولید شده مبتنی بر
فن‌آوری \lr{Intel}، که تا پیش از آن اسیر سیستم ترسناک و سیری ناپذیر ویندوز بوند،
به رایانه‌های کاملا چندکاره\LTRfootnote{Fully Multitasking}، با قابلیت استفاده
از شبکه و ایستگاهای کاری شخصی تبدیل شدند. تمام این تعییرها در سخت‌افزارها و
رایانه‌ها به دلیل زمان زیاد مورد استفاده در پردازشها و نیاز شدید استفاده از شبکه
بود.

برای به کار بردن بسیاری از سخت افزارها، شرکت‌های تولید کنند لوح‌های فشرده حاوی
سیستم‌عامل لینوکس و نرم افزارهای مورد نیاز آن را روانه بازار می‌کنند در حالی که
بسیاری از این نرم‌افزارها در شبکه جهانی قابل دسترس هستند.
سرهم بندی کردن نرم‌افزارهای مورد نیاز در سیستم‌عامل لینوکس بر اساس توزیع مورد
استفاده آن متفاوت است اما انچه که مهم است این است که عبارت \'پول هرچه را بدهی
آن را داری\' در اینجا دیگر صادق نیست.

یکی از توزیع‌های لینوکس که نام منحصر به فرد لینوکس کلاه قرمز\LTRfootnote{Red
Hat Linux} را یدک می‌کشید که توسط یک شرکت هم نام با آن توسعه می‌یافت. این توزیع
از سیستم‌عامل لینوکس کمی با دیگر توزیع‌های لینوکس موجود متفاوت بود. یکی از
مشکل‌ترین کارهای کاربران لینوکس تعیین این بود که کدام قسمت از نرم‌افزارهای موجود
در یک توزیع خاص از این سیستم‌عامل باید نصب و مورد استفاده قرار گیرد. در بسیاری
از توزیع‌های لینوکس انتخاب نرم‌افزارهای مورد نیاز برای نصب با استفاده از منوهایی
انجام می‌شد که استفاده از آن بسیار راحت بود و لینوکس کلاه قرمز نیز از این قاعده
مستثنا نبود.

اما تفاوت اصلی این توزیع با دیگر توزیع‌های لینوکس در این بود که سازندگان آن تلاش
داشتند که کاربران برای نصب یک بسته و یا نرم‌افزار کاری بیشتر از انتخاب بسته و یا
نرم‌افزار را انجام ندهند. با این وجود سیستم‌های تجاری یونیکس از سیستمی مشابه با
این نیاز استفاه می‌کرند که سیستم مدیریت بسته\LTRfootnote{Package Manger} نام
داشت. در همین راستا بسیاری از گروها در توزیع‌های متفاوت لینوکس تلاش کردند که
سیستم‌های مشابه‌ای را برای مدیریت بسته‌ها و نرم‌افزارهای ارائه دهند که هیچ کدام
به گستردگی \lr{RPM} نبود.

با گذر زمان توزیع کلاه قرمز لینوکس محبوب‌ترین توزیع لینوکس شد که امروز در دسترس
بسیاری از کاربران است. مهم‌ترین عامل موفقیت این توزیع از سیستم‌عامل لینوکس را
می‌توان \lr{RPM} معرفی کرد. گرچه در اینجا یک توصیف کوتا از این سیستم مدیریت بسته
اورده شده است اما با این وجود می‌توان کاربرد فوق تصور این روش در مدیریت
نرم‌افزارها را به سادگی حس کرد.

اما در دنیای نرم‌افزارهای رایگان یک اصل اولیه وجود دارد که عبارت است: زمانی که
یک سیستم نرم‌افزاری رایگان راهکار مناسب‌تری را ارائه می‌دهد، از آن استفاده کن.
سیستم مدیریتی \lr{RPM} نیز از این قائده مستثنا نیست. از این رو توانایی‌های
موجود در این سیستم به سرعت توجه بسیاری از کاربران و توسعه دهندگان نرم‌افزارهای
رایگان را به خود جلب کرد.

در حال حاضر علاوه بر گروه‌هایی که نرم‌افزارهای رایگان  را توسعه می‌دهند، بسیاری
از شرکت‌ها وجود دارند که محصولات تجاری خود را نیز بر اساس \lr{RPM} روانه بازار
می‌کنند. این شرکت‌ها نه تنها به این نکته دست یافته‌اند که با استفاده از این
سیستم مدیریت نرم‌افزار محصولات آنها راحت‌تر در دسترس مشتریان قرار،
بلکه ایجاد و بسته‌بندی نرم‌افزارها نیز راحت‌تر انجام خواهد گرفت.

% تعیین شود که برای ایجاد باید یک پروند برای توصیف ایجاد کرد.
گرچه هدف از این گفتار تشریح سیستم مدیریت بسته‌ها نیست اما پیش از هر چیر می‌بایست
نکات ابتدایی این سیستم مورد بررسی قرار گیرد. برای ایجاد هر بسته نرم‌افزاری از یک
پرنده استفاده می‌شود که در آن نه تنها پرونده‌های موجود در یک بسته نرم‌افزاری
بلکه روش ایجاد و نصب آنها به صورت کامل توصیف می‌شود. این پرونده یک پرونده متنی
ساده بوده و با پسوند \lr{spec} تعیین می‌شوند. در این پرونده بخش‌های متفاوتی وجود
دارد که مهم‌ترین آنها عبارت اندز از:

\begin{itemize}
  \item \lr{Preamble}
  \item \lr{prep}
  \item \lr{build}
  \item \lr{install}
  \item \lr{file}
\end{itemize}

% ساختار مورد استفاده در این بسته به صورت مقدماتی تشریح شود
\lr{Preamble} خصوصیت‌های کلی بسته مانند نام، نسخه، حق نشر و بسیاری موارد دیگر
تشریح می‌شود در حالی که دیگر قسمت‌ها به توصیف روش نصب و ایجاد پروژه خواهند
پرداخت. ابتدایی ترین بخش در این پرونده قسمت \lr{prep} است که در آن مقدمات ایجاد
نرم‌افزار و بسته بندی آن ایجاد می‌شود. در این بخش کدهای منبع موجود اماده شده و
در مسیرهای مناسب قرار می‌گیرد تا در ادامه فرآیند ترجمه و ایجاد شود. در دو بخش
دیگر که به نام‌های \lr{build} و \lr{install} ایجاد می‌شوند فرآیند ایجاد و نصب
نرم‌افزار ایجاد می‌شود. این دو بخش باید به صورت کاملا مستقل از هم در نظر گرفته
شود چرا که فرآیند نصب به روی رایانه‌های دیگر نیز اجرا می‌شود در حالی که فرآیند
ایجاد تنها به روی ماشینی اجرای می‌شود که بسته نرم‌افزاری در آن ایجاد شده است.

می‌توان گفت که \lr{file} مهم‌ترین قسمت در این پرونده است. در این قسمت تمام
رونده‌های مورد نیاز برای یک بسته و سطح دسترسی به آنها صورت کامل 
تعیین می‌شوند. در این بخش فهرست تمام پرونده‌ها و پوشه‌هایی که باید در بسته ایجاد
شده وجود داشته باشند آورده می‌شود و برای هرکدام تعیین می‌شود که سطح دسترسی
کاربران چیست.

% هدف ما و روش ایجاد این بسته تشریح شود.
مستند فنی یک سیستم نرم‌افزاری نیز بخشی از نرم‌افزار است و باید در فرآیند ایجاد
ایجاد شده و در بسته‌های مناسب قرار گیرد. از این رو باید تنظیم‌های مورد نیاز برای
ایجاد بسته مناسب مستند فنی تعیین شود. در این تنظیم نه تنها روش ایجاد مستند فنی
بلکه نصب آن نیز باید به گونه‌ای تشریح شود که مستند ایجاد شده قابل حمل و نصب به
روی دیگر رایانه‌های نیز باشد. در این میان ممکن است که یک پروژه نرم‌افزاری شامل
زیر پروژه‌های متفاوتی باشد و هر زیر پروژه به صورت مستقل مستند سازی شده باشد.

برای درک بهتر مطالب مطرح شده در این بخش یک پروژه نرم‌افزاری متن باز در نظر
گرفته شده و گام به گام تشریح شده است. این پروژه متن باز که \lr{SMath} نام دارد
یک بسته نرم افزاری است که در محاسبات اعداد بزرگ مورد استفاده قرار
می‌گیرد\cite{smath}. متن برنامه به همراه مستندات این بسته در تارنمای آن قابل
دستیابی است.

\begin{latin}
\lstset{language=TeX}  
\begin{lstlisting}[frame=single] 
http://code.p-simorgh.com/p/SMath
\end{lstlisting}
\end{latin}

این بسته نرم‌افزاری بر اساس قراردادهای تعریف شده در این کتاب ایجاد شده است و
شامل سه زیر پروژه است که عبارت اند از:

\begin{itemize}
  \item \lr{smath}
  \item \lr{smath-test}
  \item \lr{smath-test-suit}
\end{itemize}

زیر پروژه \lr{smath} شامل یک کتابخانه پویا\LTRfootnote{Dynamic Library} است که
محاسبات عددهای بزرگ را پیاده سازی می‌کند در حالی که دو بسته دیگر ارزیابی‌های این
بسته است. زیر پروژه \lr{smath-test} شامل برنامه‌های اجرایی است که در آن از
امکانات این بسته برای محاسبات استفاده شده و زیر پروژه \lr{smath-test-suit}
ارزیابی تمام امکان‌های پیاده سازی شده در این بسته است.

ساختار این پروژه نه انقدر ساده است که برای تشریح تمام موارد مورد نیاز کافی نباشد
و نه انقدر پیچیده که برای آموزش مبانی بسته بندی و خودکار سازی مستند‌های فنی
ناکار آمد باشد.

ایجاد و بسته‌بندی کردن مستند فنی پروژه می‌تواند به دو صورت تصور شود: یک بسته
کاملا مستقل و یا یک زیر بسته. هنگامی که در یک پرونده \lr{spec} روش ساخت و بسته
بندی مستند فنی یک پروژه اورده شود می‌گوییم که بسته به صورت مستقل ایجاد شده است
در حالی که اگر در یک پرونده \lr{spec} علاوه بر ایجاد خود پروژه مستند فنی نیز
ایجاد شود می‌گوییم که مستند فنی به صورت یک زیر بسته ایجاد شده است.

\subsection{یک بسته مستقل}

% اولین کار تعیین خصوصیت‌های کلی است. این خصوصیت‌ها به صورت زیر تعیین می‌شود
نخستین گامل برای ایجاد پرونده \lr{spec} در ایجاد مستند فنی تعیین خصوصیت‌های کلی
بسته مستند فنی است. خصوصیت‌های کلی هر بسته در ابتدای پرونده بسته به صورت جفت‌های
کلید مقدار تعیین می‌شود.  خصوصیت‌های کلی بسته مورد نظر ما به صورت زیر خواهد بود:

\begin{latin}
\lstset{language=TeX}  
\begin{lstlisting}[frame=single] 
Name: smath-doc
Summary: Big integer lib document
Version: 2.0
Release: 0
Group: Development/Documentation
Source: SMath-2.0.0.tar.gz
BuildArch: noarch
\end{lstlisting}
\end{latin}

برای ایجاد تماییز بین بسته‌های نرم‌افزاری و مستند فنی آنها از یک پسوند \lr{doc}
در انتهای نام بسته استفاده می‌شود. در توصیف کوتاهی که برای هر بسته پیش بینی شده
است نیز باید تعیین شود که بسته حاوی مستند فنی است و نسخه بسته نیز باید مشابه با
نسخه بسته نرم‌افزاری ایجاد شده باشد. با این روش می‌توان هموار بسته‌های
نرم‌افزاری و مستندهاای آنها را تمییز داد و تعیین کرد که مستند فنی متعلق به کدام
نسخه از بسته‌های نرم‌افزاری است.

برای جلوگیری از به هم ریختگی بسته‌های ایجاد شده بهتر است که تمام مستندها را نیز
در یک گروه قرار داد. از آنجا که مستند‌های فنی متعلق به توسعه دهندگان سیستم‌های
نرم‌افزاری است، گروه \lr{Development/Documentation} برای این مستندها در نظر
گرفته شده است. این گروه بندی بین توسعه دهندگان سیستم‌های متن باز لینوکس مرسوم
است و هیچ اجباری برای انتخاب آن وجود ندارد.

نکته‌ای که در مورد مستندهای فنی باید در نظر گرفت این است که مستندهای فنی
سیستم‌های نرم‌افزاری به هیچ معماری وابسته نیست مگر این که خود بسته نرم‌افزاری
تنها بر اساس یک معماری خاص ایجاد شده باشد. بر این اساس است که معماری مورد حمایت
در این بسته به صورت 
\lr{noarch}\footnote{واژه \lr{noarch} در اینجه به معنی مستقل از معماری در نظر
گرفته می‌شود که کوتاه شده واژه \lr{No atchitecture} است}
تعیین شده است.
 
% بازگشایی کد و ایجاد پروژه
در گام بعد باید کد منبع آماده شده تا بر اساس آن  بتوان مستند فنی ایجاد شود. از
آنجا که در فرآیندهای ایجاد در سیستم \lr{RPM} همواره یک پرونده فشرده استفاده
می‌شود که در آن کد منبع سیستم نرم‌افزاری به صورت فشرده وجود دارد، کافی است که
پرونده‌ای که شامل کد منبع است را بازگشایی کنیم. همانگونه که در کد بالا قابل
مشاهده است کد منبع مورد استفاده نیز تعیین شده است از این رو فرآیند بازگشایی با
استفاده از دستورهای خط فرمان به صورت زیر انجام خواهد شد:

\begin{latin}
\lstset{language=TeX}  
\begin{lstlisting}[frame=single]
pdir=smath-2.0.0
if [ -d $pdir ]; then
	rm -R -f $pdir
fi
mkdir -p $pdir
cd $pdir
zcat $RPM_SOURCE_DIR/SMath-2.0.0.tar.gz | tar -xvf -
\end{lstlisting}
\end{latin}

در این کد یک پوشه ایجاد شده و کد منبع در آن بازگشایی شده است. استفاده از این
تکنیک زمانی مناسب است که فرآیند ایجاد سیستم‌های نرم‌افزاری متفاوت به صورت همزمان
در حالی اجرا باشد. در این حالت با استفاده از پوشه‌های متفاوت از تداخل‌های
احتمالی میان نرم‌افزارهای جلوگیری می‌شود. اما پیش از هر کاری باید مسیرهای ایجاد
شده را حذف کرد تا تنظیم‌های مورد استفاده در فرآیند قبلی ساخت از بین برود. در
نهایت با بازگشایی متن برنامه در مسیر ایجاد شده می‌توان فرآیند ساخت را ادامه داد.

% ساخت مستند
با ایجاد مسیر مناسب و بازگشایی متن برنامه‌ها شرایط برای ایجاد مستند فنی به صورت
کامل فراهم است. در این مرحله می‌بایست تک تک مستند‌های فنی برای تمام زیر پروژه‌ها
را ایجاد کرد. قطعه برنامه زیر فرآیند ایجاد پروژه را نمایش می‌دهد: 

\begin{latin}
\lstset{language=TeX}  
\begin{lstlisting}[frame=single] 
...
mkdir -p final/doc/
cd smath
doxygen Doxygen
cd ../smath-test
doxygen Doxygen
cd ../smath-test-suit
doxygen Doxygen
\end{lstlisting}
\end{latin}

ابتدایی ترین کاری که باید انجام شود ایجاد مسیر مناسب برای مستند‌های فنی است.
همانگونه که در تعیین استانداردها تعیین شده، تمام مستند‌های فنی باید در مسیر پوشه
\lr{final/doc} ایجاد شوند از این رو پیش از ایجاد مستند فنی زیر پروژه‌ها باید از
وجود این مسیر اطمینان حاصل کرد.

در ادامه این فرآیند با ورود به پوشه هر یک از زیر پروژه‌ها مستند فنی آن ایجاد
می‌شود. مستند فنی با استفاده از دستور خط فرمان \lr{doxygen} و با استفاده
از پرونده‌های پیکره بندی هر پروژه به صورت جداگانه ایجاد می‌شود. 

\begin{note}
همواره فرض بر این است که دستور ایجاد مستند فنی در مسیر هر پروژه به صورت جداگانه
اجرا می‌شود از این رو تعیین مسیر خروجی برای هر زیر پروژه به صورت
\lr{../final/doc} تعیین می‌شود.
\end{note}

در نهایت مستند ایجاد شده باید در مسیرهای مناسب نصب شود. برای این کار کافی است که
مستندهای ایجاد شد در مسیرهای از پیش تعیین شده رو نوشت کرد. در توزیع‌های متفاوت
لینوکس مسیرهای مشخصی برای قرار دادن مستندها در نظر گرفته شده است. مسیر مستندها
در اغلب توزیع‌های لینوکس مشابه است اما گاهی در برخی از نسخه‌های متفاوت است. در
اینجا مسیرهای تعیین شده در توزیع \lr{OpenSUSE} به عنوان مسیر پیش فرض در نظر
گرفته شده و مستندها در این مسیر رونوشت شده است. از این رو برنامه نصب مستند فنی
به صورت زیر خواهد بود:

\begin{latin}
\lstset{language=TeX}  
\begin{lstlisting}[frame=single] 
%install
...
mkdir -p %{buildroot}/usr/share/doc/
cp -R final/doc/ %{buildroot}/usr/share/
\end{lstlisting}
\end{latin}

همانگونه که در این نپشته قابل مشاهده است، فرآیند نصب مستندهای فنی تنها معادل با
رو نوشت کردن مستندهای ایجاد شده در مسیرهای مناسب است. 

در نهایت باید تعیین کرد که چه پرونده‌های در بسته قرار می‌گیرند. در سیستم مدیریت
بسته \lr{RPM} تنظیم‌های پیش‌فرضی برای پرونده‌های مستند در نظر گرفته شده است از
این رو نیازی به تعیین تنظیم خاص در این قسمت نیست و تنها کافی است که فهرست
پرونده‌ها را تعیین کرد. برای نمونه در اینجا پرونده مستندها به صورت زیر در بسته
جای می‌گیرد:

\begin{latin}
\lstset{language=TeX}  
\begin{lstlisting}[frame=single] 
%docdir /usr/share/doc
/usr/share/doc
\end{lstlisting}
\end{latin}

همانگونه که در برنامه قابل مشاهد است، تنها مسیر مستند در فهرست پرونده‌ها
قرار گرفت است. از انجا که تمام مستندهای ایجاد شده همگی در این مسیر نصب (یا رو
نوشت شده است) تنها کافی است که این مسیر به همراه تمام پرونده‌های موجود را در
بسته ایجاد شده قرار داد. در نهایت پرونده \lr{spec} برای مستند فنی این بسته به
صورت زیر خواهد بود:


\begin{latin}
\lstset{language=TeX}  
\begin{lstlisting}[frame=single] 
%define name smath-doc
%define version 2.0
%define release 0

Name: %{name}
Summary: Big integer lib document
Version: %{version}
Release: %{release}
License: GPL
Group: Development/Lib
Vendor: Simorgh 
Packager: Mostafa Barmshory <mostafa.barmshory@p-simorgh.com>
URL: http://code.p-simorgh.com/index.php/p/smath/
Source: SMath-%{version}.%{release}.tar.gz
BuildArch: noarch

%description
It is simple big integer lib. 

%prep
pdir=SMath-%{version}.%{release}
if [ -d $pdir ]; then
	rm -R -f $pdir
fi
mkdir $pdir
cd $pdir
zcat $RPM_SOURCE_DIR/SMath-%{version}.%{release}.tar.gz | tar -xvf -

%build
pdir=SMath-%{version}.%{release}
cd $pdir
mkdir -p final/doc/
cd smath
doxygen Doxygen
cd ../smath-test
doxygen Doxygen
cd ../smath-test-suit
doxygen Doxygen

%install
pdir=SMath-%{version}.%{release}
cd $pdir
mkdir -p %{buildroot}/usr/share/doc/
cp -R final/doc/ %{buildroot}/usr/share/

%files
%docdir /usr/share/doc
/usr/share/doc
\end{lstlisting}
\end{latin}

% نصب 
در نهایت با نصب بسته ایجاد شده مستندهای فنی نرم‌افزار در سیستم نصب شده و قابل
استفاده می‌باشد. ایجاد بسته با استفاده از دستور \lr{rpmbuild} انجام می‌شود.
ایجاد بسته مورد نظر به صورت زیر خواهد بود.

\begin{latin}
\lstset{language=TeX}  
\begin{lstlisting}[frame=single] 
rpmbuild -ba smath2.spec
\end{lstlisting}
\end{latin}

\subsection{زیر بسته}

نمی‌توان مستندهای تکنیکی را مستقل از خود بسته‌های نرم‌افزاری در نظر گرفت از این
رو ایجاد مستندهای فنی نیز می‌تواند توام با ایجاد خود بسته‌های نرم‌افزاری انجام
شود. در این حالت از روش زیر بسته‌ها استفاده می شود.

در این روش یک بسته به عنوان بسته اصلی در نظر گرفته می‌شود و دیگر بسته‌ها به صورت
زیر بسته‌های بسته اصلی در نظر گرفته می‌شود. برای نمونه بسته مستند فنی می‌تواند
به عنوان زیر بسته‌ای از بسته اصلی \lr{smath} در نظر گرفته شود.

سیستم مدیریت بسته \lr{RPM} راهکارهای مناسبی را برای ایجاد و مدیریت بسته‌ها و زیر
بسته‌ها ایجاد کرده است. در اینجا نیز برای هر زیر بسته دسته‌ای از اطلاعات کلی
وجود دارد که به صورت زوج‌های مقدار کلید تعریف می‌شوند با این تفاوت که اطلاعات
کلی هر زیر بسته با استفاده از برچسب \lr{package} تعیین می شود.

تکه برنامه زیر خصوصیت‌های کلی مورد نظر برای بسته مستند فنی را تعیین کرده است:

\begin{latin}
\lstset{language=TeX}  
\begin{lstlisting}[frame=single] 
%package doc
Summary: Big integer lib document.
Group: Development/Documentation
BuildArch: noarch
%description doc
It is simple big integer lib. Documentation is used for developer
\end{lstlisting}
\end{latin}

نام هر زیر بسته که بعد از برچسب \lr{package} اورده می‌شود به عنوان یک پسوند از
عنوان بسته اصلی در نظر گرفته می‌شود از این رو نام بسته مستند فنی در اینجا نیز
به صورت \lr{smath-doc} خواهد بود.

تمام خصوصیت‌های تعریف شده برای بسته اصلی در زیر بسته‌ها نیز به ارث می‌رسد و در
صورت نیاز می‌توان آنها را باز نویسی کرد. در اینجا نیز نسخه و بسیار از خصوصیت‌های
دیگر با بسته اصلی یکی است از این رو نیازی به تعیین این خصوصیت‌ها نیست. 

گرچه تمام خصوصیت‌های بسته اصلی توسط زیر بسته‌ها نیز به ارث می‌رسد اما برای هر
زیر بسته می‌بایست خصوصیت‌هایی مانند گروه، و توصیف بسته را باز نویسی کرد. از انجا
که بسته ایجاد شده یک بسته شامل مستندهای فنی است نه تنها توصیف‌های این زیر بسته
باید بیانگر این مطلب باشد بلکه گروه بسته نیز باید به صورت مناسب تعیین شود. از
این رو گروه این زیر بسته به صورت \lr{Development/Documentation} در نظر گرفته شده
است. علاوه بر این ممکن است که بسته اصلی بر اساس معماری خاصی ایجاد شود در حالی که
مستند فنی سیستم به هیچ معماری وابسته نیست. برای تعیین این نکته خصوصیت معماری
مورد استفاده نیز به صورت زیر بازنویسی شده است:

\begin{latin}
\lstset{language=TeX}  
\begin{lstlisting}
BuildArch: noarch
\end{lstlisting}
\end{latin}

گرچه هر زیر بسته می‌تواند از خصوصیت‌های منحصر به فرد خود استفاده کنند اما کدهای
منبع مورد استفاده میان تمام انها مشترک است. بدیهی است  فرآیند آماده سازی کدهای
منبع نیز در تمام زیر بسته‌ها نیز مشترک باشد. زیر بسته مستند فنی نیز از این قائد
مستثنی نیست و مبتنی بر کد منبع کل سیستم ایجاد خواهد شد.

\begin{note}
فرآیند ایجاد کد منبع مشابه به حالی است که در آن از زیر بسته‌ها استفاده نمی‌شود
از این رو در اینجا به آن پرداخته نشده است.
\end{note}

برخلاف اماده سازی کد منبع سیستم، فرآیند ایجاد آن بسیار پیچیده خواهد بود. در این
فرایند نه تنها باید برنامه‌های ایجاد شده به صورت کامل ترجمه، ارزیابی و پیونده
زده شوند، بلکه باید مستندهای فنی و سایر موارد مورد لزوم دیگر نیز ایجاد شود.

ایجاد دیگر بخش‌های پروژه در اهداف این کتاب نمی‌گنجد از این رو تنها روش ایجاد
مستند فنی سیستم مورد نظر است. ایجاد مستند فنی نیز بر اساس روشهای مطرح شده در
بخش قبل خواهد بود. کد مورد نیاز برای ایجاد مستند فنی به صورت زیر خواهد بود.

\begin{latin}
\lstset{language=TeX}  
\begin{lstlisting}[frame=single] 
%build
...
mkdir -p final/doc/
cd smath
doxygen Doxygen
cd ../smath-test
doxygen Doxygen
cd ../smath-test-suit
doxygen Doxygen
\end{lstlisting}
\end{latin}

برخلاف آماده سازی و ایجاد پروژه، که در تمام زیر پروژه‌ها مشترک است، فرآیند نصب
برای هر زیر بسته به صورت مستقل اجرا می‌شود. دور از انتظار نیست که فرآیند نصب
بسته مستند فنی با نمونه‌ای که مستند فنی به صورت مستقل ایجاد می‌شود تفاوتی نداشته
باشد. گرچه فرآیند نصب مستند فنی به صورت مستقل و در بخشی جداگانه ایجاد می‌شود اما
در اینجا نیز مستندهای ایجاد شده تنها در مسیرهای مناسب رو نوشت خواهد شد. فرآیند
نصب بسته مستند فنی نیز به صورت زیر خواهد بود.

\begin{latin}
\lstset{language=TeX}  
\begin{lstlisting}[frame=single] 
%install doc
...
mkdir -p %{buildroot}/usr/share/doc/
cp -R final/doc/ %{buildroot}/usr/share/
\end{lstlisting}
\end{latin}

بخش دیگری که برای هر بسته به صورت مستقل ایجاد می‌شود فهرست پرونده‌های موجود در
هر زیر بسته است. از آنجا که هر زیر بسته ممکن است بر اساس طبیعت خود فهرست متفاوتی
از پرونده‌های پروژه را داشته باشد، مستند فنی نیز شامل تمام پرونده‌های ایجاد شده
در فرآیند مستند سازی خواهد بود. کد زیر تعیین فهرست مستند فنی برای زیر بسته مستند
فنی را نمایش می‌دهد.

\begin{latin}
\lstset{language=TeX}  
\begin{lstlisting}[frame=single] 
%files doc
%docdir /usr/share/doc
/usr/share/doc
\end{lstlisting}
\end{latin}

در نهایت با فراخوانی دستور \lr{rpmbuild} نه تنها تمام بسته‌های نرم‌افزاری بلکه
بسته مستند فنی نیز ایجاد می‌شود. این بسته وابسته به هیچ معماری نبوده و به سادگی
به روی سیستم‌های متفاوت نصب خواهد شد.


